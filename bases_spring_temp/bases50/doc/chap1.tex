%
\newpage
\noindent
{\bf ~~~LONG WRITE-UP}
\bigskip
\par
The Monte Carlo integration and event generation package {\small BASES/ SPRING V1.0}
$\lbrack 1 \rbrack$ has been upgraded so as to generate events with 50 independent
variables, and  to integrate functions with  an
alternating sign. 
Its ability to integrate real functions with indefinite sign is
 found to be useful in the numerical evaluation of interference effects among
 various amplitudes.
Besides these changes, its program structure has been completely reformed.  
\par \vskip 1.0cm
\leftline{\bf 1. Introduction }
\bigskip
\par
The numerical integration and event generation package {\small BASES/SPRING
 V1.0}  has been applied to many elementary processes $\lbrack
2 \rbrack$.
In recent years very high energy $e^+e^-$ colliders, {\small JLC ( KEK )}, {\small NLC 
( SLAC )}, and {\small CLIC ( CERN )}, are being studied as future plans.
At the energies of these machines, the number of final state particles in
some elementary processes is so many that we cannot apply the version {\small V1.0} 
directly because of its limited number of dimensions ( ten at maximum ).
Extension of the number of dimensions is easy for the integration package {\small
BASES}, but it is not for the event generation package {\small SPRING}. 
Before
discussing this point, it is useful to describe the algorithm of the version
{\small  V1.0} briefly.

{\small BASES} integrates a function by means of the importance and stratified 
sampling method. 
\par\noindent
In order to realize the importance sampling each variable axis is divided into
$N_{region}$ {\it regions}, each of which is subdivided into $m$ {\it subregions}.
In total each variable axis has $N_g$ ( = $N_{region}$ $\times$ $m$ ) subregions. 
The numbers $N_{region}$ and $m$ are determined from the dimension of integral $N_{dim}$
automatically by {\small BASES}.
The full integral volume is covered by a {\it grid} of ${N_g}^{N_{dim}}$
subregions. 
We call a subspace of the
integral volume a {\it hypercube} which is spanned by the regions of each variable axis.
\par
In each hypercube sample points of a definite number $N_{sample}$ are taken and
the integral and its variance are evaluated.
The estimate of integral and its variance are calculated by summing up results of all
hypercubes. 
We call this process an {\it iteration}.
A cumulative estimate and its error are calculated from the results of all
iterations statistically.
\par
Execusion of {\small BASES} consists of the grid optimization and the integration
steps.
\par\noindent
At the first iteration of the former step the grid is uniformly defined for each
variable axis.   After each iteration the grid is adjusted so as to make the sizes of
subregions the narrower at the parts with the larger function value and the wider at
the parts with the smaller one. In this way a suited grid to the integrand is
obtained.  
\par\noindent
In the integration step, the probability to select each hypercube and
the maximum value of the function in it are calculated as well as the estimate of
integral with the frozen grid determined in the former step.
\par\noindent
In {\small SPRING}, a hypercube is sampled according to its probability,
calculated by {\small BASES}, and a point in the hypercube is selected by
sample-and-reject method for which the maximum value of the function is used.
If the grid is sufficiently optimized,  a high efficiency of generating
a set of values of the independent variables (i.e.~event generation) can be
achieved even by such a simple method.
\par
The difficulty in the straightforward extension of the number of dimensions in {\small
BASES / SPRING V1.0} can be easily seen by considering a 25 dimensional case
(i.e.~25 independent variables). 
Even though we suppose to have only two regions for each variable
axis, there are $N_{cube}$ = $2^{25}$ hypercubes in total. 
The calculation of such
a huge number $N_{cube}$ of probabilities requires much {\small CPU} time and their
storage space is also too large. 
It seems to be unrealistic.
\par
By assumming, however, an additional condition to the integrand,
this difficulty can be avoided. 
The algorithm of the new version {\small BASES/SPRING
V5.1} is described in the next section, and its program structure and usage are
shown in section 3. In section 4 we will discuss its test run.