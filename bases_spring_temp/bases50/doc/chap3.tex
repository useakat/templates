\vskip 1.0cm
\leftline{\bf 3. Program Structure and usage }
\bigskip
In  the old version, users must prepare many subprograms,
e.g.~{\tt USERIN}, {\tt FUNC}, {\tt USR\-OUT}, {\tt SP\-INIT}, {\tt SP\-EVNT},
{\tt SP\-TERM} etc.~according to the specification of each subprogram.
This complexity seems to give a pedestrian a high threshold to use this packge.
Furthermore, the program flow is completely controlled by the {\small BASES/SPRING}
package, and the numerical integration and the event generation are designed to be
successively performed by two separate procedures.  This makes it less
flexible particularly in the event generation, to pass generated events to
detector simulation and data analysis. In order to dissolve these
inconvieniencies, the program structure has been completely reformed.
 \par
\bigskip
\leftline{\bf 3.1 Program components }
\par\medskip
When we study an elementary process, we first calculate an effective cross
section by integrating the differential cross section with some kinematical cuts
by {\tt BASES}, generate events of four momentum vectors of final state particles
by {\tt SPRING}, pass through a detector simulation program to simulate the
experimental data, and then analyze the resultant
simulation data to compare with real data. 
\par\noindent
If the process has  a few particles in the final states, the numerical
integration and the event generation do not take so much computing time that it
is better to carry out the simulation study in one procedure for simplicity.
If the process has many final state particles, however, the
numerical integration may take much computing time.  It is then recommended to
separate the integration part from the other parts consisting of the event
generation and the detector simulation, etc.
\par\noindent 
Since, in the present version the numerical integration and the event generation
are carried out by calling subroutines {\tt BASES} and {\tt SPRING}, respectively,
the simulation study can be performed either in one procedure or in separate two
or more procedures.
In any case users are required to prepare main and function programs.
In addition to the subroutines {\tt BASES} and {\tt SPRING}, a set of subroutines,
described below, are prepared for use.
It should be noticed that all the real variables used in the system are defined to be
of double precision.
\par
{\small
\begin{description}
\item{\tt BSINIT} : To set the {\small BASES/SPRING} parameters equal to defaults. 
\item{\tt BASES} : To make numerical integration and to prepare the probability
information for event generation. 
\item{\tt BSINFO} : To print a parameter list and convergency 
behavior of the integration. 
\item{\tt BHPLOT} : To print histograms and scatter plots, taken 
in the integration. 
\item{\tt SPRING} : To generate an event with weight one. 
\item{\tt SPINFO} : To print statistical information about the 
event generation.
\item{\tt SHPLOT} : To print histograms and scatter plots, taken  
in the event generation.
\item{\tt BSWRIT} : To save the probability information into a file.
\item{\tt BSREAD} : To read the probability information from a file.
\end{description}}
 \par
\par\noindent
An example of outputs from {\tt BSINFO}, {\tt BHPLOT}, {\tt SPINFO}, and {\tt
SHPLOT} is shown and explained for a test run output in
section 4.
\par
  The program flow is fully controlled by the main program written by users.
Its specification is described in the next subsection together with the usage of
these subroutines. 
  
\par  
\bigskip 
\leftline{\bf 3.2 Main program }  
\par\medskip
The main program can be divided into the initialization part of {\small
BASES/SPRING}, the numerical integration part, and the
event generation part.
A list of the main program for the test run is given in Fig.1 as an example.
\par\medskip
\noindent
(1) {\bf Initialization of BASES/SPRING}  
\par\smallskip
At the beginning, {\tt BSINIT} is called to set the parameters for {\small 
BASES/SPRING} equal to default values and then the parameters for the integration and
event generation are to be initialized to their proper values, which are passed
to the system through the labeled commons {\tt BPARM1} and {\tt BPARM2}. 
All real parameters in these two labeled commons are to be given with double
precision. The content of common {\small\tt BPARM1} is as follows:      
{\small \begin{verbatim}
 COMMON /BPARM1/ XL(50),XU(50),NDIM, 
.                NWILD,IG(50),NCALL 
\end{verbatim}}
\par
{\small
\begin{description}
\item{\tt NDIM}~~: The number of independent variables ( dimension
of integral) up to 50.
\item{\tt NWILD}~~: The number of the wild variables up
to 15.
\item{\tt XL({\it i}), XL({\it i})}~~:The lower and the upper limits of i-th
integral variable.  Notice 
that the first {\tt NWILD} variables must be the wild variables in each array.
\item{\tt IG({\it i})}~~: The flag to switch the sampling method for {\it i }-th
variable.\\
( 0 / others ) = ( uniform / improtance sampling)
\item{\tt NCALL}~~: The number of sample points per iteration.
\end{description}}
\par\noindent
The true number of sample points $N_{call}^{real}$ differs from a given number {\tt
NCALL}, which is automatically determined by the following algorithm.
The number of regions per variable $N_{region}$ is determend as the maximum number
which satisfies the two inequalities:
\[
N_{region}  =  ({ N_{call} \over 2 })^{ 1 \over N_{wild}}  \leq 25
\]
\par\noindent
and\par
\[
N_{region}^{N_{wild}}  <  32768.
\]
\par\noindent
The number of hypercubes is given by $N_{cube}$ = ${N_{region}}^{N_{wild}}$, then the
number of sample points per cube is $N_{sample} = N_{call}/N_{cube}$.
Since the number $N_{sample}$ is an integer, the calculated number $N_{call}^{(real)}
= N_{sample}\times N_{cube}$ may differ from the given number $N_{call}^{(given)}$
(= {\tt NCALL}). \par
 The content of common {\tt BPARM2} is as follows:
 
{\small \begin{verbatim}
 COMMON /BPARM2/ ACC1,ACC2,ITMX1,ITMX2
\end{verbatim}}
\par
{\small
\begin{description}
\item{\tt ACC1}~~:  The desired accuracy of integration for the grid
optimization step in units of percent.
\item{\tt ACC2}~~:  The desired accuracy of integration for the integration
step in units of percent.
\item{\tt ITMX1}~~:  The number of iterations for the grid optimization
step.
\item{\tt ITMX2}~~:  The number of iterations for the integration step.
\end{description}}
\par\noindent
In addition to the initialization of the parameters in these two labeled commons,
parameters and constants, like center of mass energy or beam energy, 
kinematical cut values, the fine structure constant and
the numerical value of $\pi$, etc., should be set and be passed to the function
program {\tt FUNC} through some labeled commons to calculate the differential cross
section.
\par
When we make histograms and scatter plots of some variables, their
initialization should be done here. 
Usage of the histogram package proper the system is described in subsection 3.4. 
\par\medskip
\noindent
(2) {\bf Numerical integration.}
\par\smallskip
In the grid optimization step, sizes of the subregions for each variable are
optimized to the behavior of integrand, iteration by iteration, starting from a
uniform size of subregions, as briefly described in section 1. The algorithm for the
grid  optimization is identical to that of the
old version and VEGAS, whose more detailed description is
found in references $\lbrack 1 \rbrack$ and $\lbrack 3 \rbrack$.  
\par\noindent
In the integration step, by fixing the grid determined in the grid
optimization step a cumulative estimate of the integral is calculated as well as the
probability information for each hypercube spanned by the wild variables.
\par\noindent
 The both steps are terminated when the estimated accuracy of the integral reaches a
required value or the number of iterations exceeds a given number. 
\par
The estimate of the
integral and its error etc.~of the integration step are returned to users in the
arguments of the subroutine {\tt BASES} as follows:
{\small
\begin{verbatim}
 CALL BASES( FUNC, ESTIM, ERROR, 
.                  CTIME, IT1,   IT2 )
\end{verbatim}}
\par\noindent
{\small
\begin{description}
\item{\tt FUNC}~: The name of a function program, whose specification is
described in subsection 3.3.  
\item{\tt ESTIM}~: A cumulative estimate of the integral.  
\item{\tt ERROR}~: The standard deviation of the estimate of the integral. 
\item{\tt CTIME}~: The computing time used by the integration step in seconds. 
\item{\tt IT1}~: The number of iterations made in the grid optimization
step.  
\item{\tt IT2}~: The number of iterations made in the
integration step. 
\end{description}} 
\par
By calling the subroutine {\tt BSINFO}, tables of the convergency behavior both
for the grid optimization and the integration steps are printed on a logical
unit
 {\small\tt LU}.
For example, the following statement:
\par
{\small\begin{verbatim}
  LU   = 20
  CALL BSINFO( LU )
\end{verbatim}}
\par\noindent
print these tables on a file which
should be opened in the initialization step (1).
The contents of the tables are described in subsection 4.2.
\par  
When histograms and scatter plots are made, they
can be printed on a logical unit {\small\tt LU } by  
\par
{\small\begin{verbatim}
  CALL BHPLOT( LU ).
\end{verbatim}}
\par
Subroutines {\tt BSINFO} and {\tt BHPLOT} need not to be called, unless outputs
from them are neccessary.   \par
If the numerical integration and the event generation are saparately performed
by different procedures, the probability information made by {\tt BASES} should
be saved on a file by calling {\tt BSWRIT} , for instance, as follows:
\par
{\small\begin{verbatim}
  LN  = 23
  CALL BSWRIT( LN ).
\end{verbatim}} 
The file allocated to the logical unit {\tt LN} must be opened at the
initialization step (1).
Futhermore, at the beginning of the event generation program and
just after the initialiation of BASES / SPRING described above, subroutine
{\tt BSREAD} should be called to read the probability information from
the file as follows:
\par
{\small\begin{verbatim}
  LN  = 23
  CALL BSREAD( LN ),
\end{verbatim}}
\par\noindent
then the event generation can be carried out. 
\par\medskip 
\noindent  
(3) {\bf Event generation} 
\par\smallskip
After the integration events can be immediately generated by calling {\tt
SPRING}. 
For each call of subroutine {\tt SPRING} a set of values of independent
variables is generated with weight one.
The calling sequence of {\tt SPRING} is as follows:
\par
{\small
\begin{verbatim}
 CALL SPRING( FUNC, MXTRY )
\end{verbatim}}
\par
{\small
\begin{description}
\item{\tt FUNC} : The name of a function program.
\item{\tt MXTRY} : The maximum number of trials to generate one event.
\end{description}}
\par\noindent
At the first call of {\tt SPRING} a cumulative
probability distribution over all the hypercubes is calculated from the
differential one, prepared by {\tt BASES}.
This cumulative distribution is used to sample a hypercube according to
its probability. 
\par
After a hypercube is sampled in this way, a point is selected in this hypercube by
the importance sampling method and is examined whether it is accepted or not. 
When this point is not accepted, another point is sampled in the same hypercube
and tested again.
If the grid is not optimized enough, there may be a possibility to get into
an infinite loop of generation.
To avoid this occurence the number of trials to get an event is limited to
{\tt MXTRY } whose recommended number is 50.
The case where an event could not be generated even by {\tt MXTRY} trials is called
{\it
 mis-generation}.
If the number of mis-generations exceeds {\tt MXTRY}, a warn\-ing message
is printed. 
\par
The event generation by {\tt SPRING} determines only a set of values of
independent variables of the function parogram.
Users are required to calculate the four-momentum vectors from this set
of variables for themselves.
A typical structure of the event generation step is as follows:
\par
\begin{center}{\footnotesize
\begin{verbatim}
    MXTRY  = 50
    ISPRNG = 1
    DO 100 NEV = 1, 10000
      CALL SPRING(FUNC,ISPRNG,MXTRY)
            . . . .
      Calculate four momentum vectors
            . . . .
100 CONTIINUE
\end{verbatim}}\end{center}
\par\noindent
Since in the function program the kinematics of the process is calculated,
it is better to use the result of kinematics for calculating the four-momentum
vectors. 
\par
After the event generation, statistical information about the generation can be
obtained by calling {\tt SPINFO} as follows:
\par
{\small
\begin{verbatim}
  CALL SPINFO( LU ) ,
\end{verbatim}}
\par\noindent
where {\tt LU} is the logical unit of an output device.
Histograms and scatter plots can be also printed on the logical unit {\tt LU}
by calling {\tt SHPLOT}. 
\par
{\small
\begin{verbatim}
  CALL SHPLOT( LU ).
\end{verbatim}}
\par \bigskip  
\leftline{\bf 3.3 Function program }  
\par\medskip
In the function program the value of the integrand is calculated at the sample point
fed by {\tt BASES}.
 A set of numerical values of integral variables is passed through
the argument of the function program.
A typical structure of function program is as follows:
\par\begin{center}
{\footnotesize\begin{verbatim}
  DOUBLE PRECISION FUNCTION FUNC(X)
  REAL*8 X(2)
  FUNC  =  0.0D0

 ... Calculation of kinematics ...
  IF(the point is outside of
     the kinematical boundary) RETURN

  FUNC = is calculated
       from X(i) for i = 1, 2

  CALL XHFILL(ID, V, FUNC)
  CALL DHFILL(ID,VX, VY, FUNC )

  RETURN
  END
\end{verbatim}}\end{center}
\par\noindent
A recipe for writing a function program is as follows:
{\small\begin{description}
\item{1)} Calculate the kinematical variables by which the differential cross
section is described from the integral variables, {\tt X({\it i})} for {\it i}
= 1 to {\tt NDIM}.
\item{2)} If the point fed by {\tt BASES} is found to be outside the
kinematical boundary, set the value of the function equal to zero and return.
\item{3)} If the point is inside the kinematical boundary, calculate the
numerical value of the function at the point and return  it as the function value.
\item{4)} If a histogram is required, call subroutine {\tt XHFILL} once.
\item{5)} If a scatter plot is required, call {\tt DHFILL} once.
\end{description}}
\par\noindent
An example of {\tt FUNC} for the process $e^+e^-$ $\rightarrow$ $\nu_{\mu}
\bar{\nu}_{\mu} \gamma$ is given in Fig.2.
\par \bigskip  
\leftline{\bf 3.4 Histogram package }   
\par\medskip 
{\tt BASES} has a proper histogram package, by which histograms and scatter plots of
any variables can be made.
A characteristic of this package is that there are two kinds of histograms, the
one is called ``original histograms'' and another is ``additional histograms''.
\par
The original histogram is quite useful to check whether the frequency
distribution of generated events really reproduces the distribution calculated in
the integration stage. 
On the other hand the additional histogram prints only the frquency distribution of
 the generated events. 
A detailed description of these two kinds of hitograms is found in subsection
4.2.
\par 
The maximum numbers of histograms and scatter plots are 50 for each.
If the initialization routine {\tt XHINIT} ( or {\tt DHINIT} ) is called more than 50
times, subsequent calls are neglected.
Usage of this histogram package  is as follows:
\par
{\small\begin{description}
\item{1)} To initialize histograms and scatter plots the following routines are
to be called:
\begin{verbatim}
 CALL XHINIT(ID#,
.     lower_limit,
.     upper_limit,# of bins, 
.    'Title of histogram')
\end{verbatim}
and
\begin{verbatim}
 CALL DHINIT(ID#,
.     x-lower_limit,
.     x-upper_limit,# of x bins,
.     y-lower_limit,
.     y-upper_limit,# of y bins,
.    'Title of scatter plot'),
\end{verbatim}
respectively.
\item{2)} To fill the histograms and scatter plots the following routines are to be
called in {\tt FUNC}:
\begin{verbatim}
 CALL XHFILL(ID#,V,FUNC)
\end{verbatim}
and
\begin{verbatim}
 CALL DHFILL(ID#,Vx,Vy,FUNC)
\end{verbatim}
respectively.
\item{3)} To print the resultant histograms and scatter plots taken in the
integration,
\begin{verbatim}
 CALL BHPLOT(LU)
\end{verbatim}
\noindent
where {\tt LU} is a logical unit number.
\item{4)} To print the resultant histograms and scatter plots taken in the
event generation,
\begin{verbatim}
 CALL SHPLOT(LU)
\end{verbatim}
\noindent
where {\tt LU} is a logical unit number.
\end{description}}
\par\noindent
Examples of histograms and scatter plots printed by {\tt BHPLOT} and {\tt SHPLOT}
are shown in the test run output.
