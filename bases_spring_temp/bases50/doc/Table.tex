\onecolumn
\begin{center}
\begin{tabular}{|c|c|c|c|c|c|c|} \hline
\multicolumn{5}{|c|}{ Numerical Integration } & \multicolumn{2}{c|}{ Event Generation } \\ 
\hline Case & Estimate ( error ) & CPU time & It-1 & It-2 & Efficency & CPU time \\ \hline
  1)  & 4.528033 ($\pm$0.002230)~$\times 10^{-2}$ & 3.43 sec & 10 & 26 & 75.8 \% & 0.40 sec \\
\hline
  2)  & 4.527550 ($\pm$0.002255)~$\times 10^{-2}$ & 3.59 sec & 10 & 27 & 69.4 \% & 0.41 sec \\
\hline
  3)  & 4.521658 ($\pm$0.006430)~$\times 10^{-2}$ & 10.67 sec & 10 & 100 & 11.6 \% & 1.88 sec
\\ \hline \end{tabular}
\par
\vskip 1.0cm
\begin{minipage}{12.0cm}
Table~1. The results for three cases of integration parameters: case 1)
is by the old algorithm. Cases 2) and 3) are by the
new version with the importance and uniform samplings for the gentle
variable, respectively. It-1 and It-2 are the numbers of iterations
for the grid optimization and the integration steps, respectively.
The generation efficiency is defined to be the ratio of the number of accepted events to that
 of trials.
The {\small CPU} time for generation is the computing time consumed to generate 10k
events and calculate their four-momentum vectors. 
\end{minipage}
\end{center}
\par
\vskip 4.0cm
\centerline{\Large\bf Figure Captions}
\par\vskip 1.0cm
\begin{center}
\begin{minipage}{8.0cm}
\begin{description}
\item{Figure 1.} Main program of the test run.
\item{Figure 2.} Function program of the test run.
\end{description}
\end{minipage}
\end{center}
\par
