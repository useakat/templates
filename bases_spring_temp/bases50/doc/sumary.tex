
%%%%%%%%%%%%%%%%%%%
{\footnotesize
%%%%%%%%%%%%%%%%%%%
\noindent
{\it Title of program}: BASES/SPRING V5.1
\par
\smallskip
\noindent
{\it Program obtainable from}:~CPC Program Library, Queen's University
of Belfast, N.Ire\-land ( see application form in this issue )
\par
\smallskip
\noindent
{\it Computer for which the program is designed and others on which it is 
operable}:~HP750, FACOM, HITAC, IBM, VAX and others with a FORTRAN77 compiler.
\par
\smallskip
\noindent
{\it Computer}: HP750, FACOM-M780; ~{\it Installation}: National Laboratory for
High Energy Physics (KEK), Tsuku\-ba, Ibaraki, Japan
\par
\smallskip
\noindent
{\it Operating system}:~UNIX, OSIV/F4 MSP
\par
\smallskip
\noindent
{\it Programming language used}: FORTRAN77
\par
\smallskip
\noindent
{\it High speed storage required}: 392 Kwords 
\par
\smallskip
\noindent
{\it No.~of bits in a word}: 32
\par
\smallskip
\noindent
{\it Peripherals used}: disc file and printer for output
\par
\smallskip
\noindent
{\it No.~of cards in combined program and test deck}: 2750
\par
\smallskip
\noindent
{\it Card image code}: ASCII
\par
\smallskip
\noindent
{\it Key words}: Multi-dimensional integration, Monte Carlo simulation,
 event generation and four momentum vector generation.
\par
\smallskip
\noindent
{\it Nature of physical problem}
\par
\noindent
The previous version of the numerical integration and event generation package
BASES/SPRING has been useful to obtain  total cross sections and to generate 
events of elementary processes in high energy physics. It is applicable to
processes with up to ten independent variables. 
In order to study, for example, $e^+e^-$ physics at much higher
energies, we often need more than ten independent variables to
describe  processes of our interest. 
As far as  the numerical integration by BASES is concerned, it is
easy to extend the dimension of integral. 
However, the event generation requires a huge memory space with the previous 
generation algorithm of SPRING. \par
\smallskip
\noindent
{\it Method of solution}
\par
\noindent
BASES/SPRING is suited for the integration and generation of a very
singular function.
The number of those independent variables which give the function a
singular behavior is usually small.
Then, if we divide the subspace spanned by these singular variables into
hypercubes, the number of hypercubes becomes not too large.
Applying the previous BASES / SPRING algorithm only to this subspace and
handling the other variables as additional integral variables, we could
extend the dimension of integral and event generation up to 50 variables.
\par
\smallskip
\noindent
{\it Typical running time}
\par
\noindent
The running time is essebtially determined by the complexity of the function
program which gives the differential cross section of an elementary process.
If we take the process ${e^+e^-}$ $\rightarrow$ ${\nu \bar{\nu} \gamma }$
as an example, the two dimensional integration and the generation of 10000
events require 4.4 seconds total on a FACOM M780
computer.
}