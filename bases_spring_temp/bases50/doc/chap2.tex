\vskip 1.0cm
\def\d{{\rm d}}
\leftline{\bf 2. Algorithm of Version 5.1}
\bigskip
\par
In usual practice the function of our interest has only a limited
number of variables that cause singular behavior. 
We call such kind of variables {\it wild variables}, and rest  {\it gentle
variables}, on which the function depends only weakly.  If we divide only the
subspace spanned by these wild variables into hypercubes, the number of
hypercubes is not too large.  Thus, by applying the {\small BASES/SPRING V1.0}
algorithm only to this subspace and by handling the gentle variables as
additional integral variables, we can overcome the difficulty in the event
generation of higher dimensions.

In order to see the difference between the old and the new algorithms, we
consider the following integral of three dimensions: 
\[
 I = \int\nolimits_0^1 f(x,y,z)  \d x \d y \d z.
\]
In the old algorithm, the estimate of integral is given by
\[
 I \simeq \sum_{j=1}^{N_{cube}} {1\over{N_{sample}}}
 \sum_{i=1}^{N_{sample}}{ f(x_i^j, y_i^j, z_i^j) \over p(x_i^j, y_i^j, z_i^j)},
\]
\noindent
where $x_i^j$, $y_i^j$ and $z_i^j$ indicate $i$-th sample point in $j$-th
hypercube,  $p(x_i^j, y_i^j, z_i^j)$ is the probability density at the point for
the importance sampling, and $N_{sample}$ is the number of sample points in each
hypercube. \par 
On the other hand, if we take the new algorithm and suppose the variables $x$ 
and $y$ are wild and $z$ is gentle, the estimate is given by 
\[
 I \simeq \sum_{j=1}^{N_{cube}} {1\over{N_{sample}}}
 \sum_{i=1}^{N_{sample}}{ f(x_i^j, y_i^j, z_i) \over p(x_i^j, y_i^j, z_i)},
\]
\noindent
where $z$ is sampled in the full range of $z$ variable.
Since the importance sampling takes place even for the gentle variable $z$, the
numerical integration converges fast enough.  An efficient event generation is
guaranteed if the function 
\[
F^j( z ) =  \int\nolimits_{j-th hypercubes} f( x^j, y^j, z )  \d x^j \d y^j
\]
has similar dependences on $z$ for all j.

The numbers of hypercubes are $N_{region}^3$ and $N_{region}^2$ in the old and the
new algorithms, respectively, where $N_{region}$ is the number of regions per
variable. Generally speaking, if $N_{dim}$ and $N_{wild}$ are the numbers of
independent variables and the wild variables ( $N_{wild}$ $\le$ $N_{dim}$ ),
respectively, the number of hypercubes $N_{cube}$ is given by
\begin{eqnarray*}
 N_{cube}    & =  {N_{region}}^{N_{dim}} &{\rm (old)}\cr
      ~      & =  {N_{region}}^{N_{wild}} &{\rm (new).}
\end{eqnarray*}
In version 5.1, the maximum number of wild variables is equal to 15.
