\vskip 1.0cm
\leftline{\bf 5. Conlusion}
\par\medskip
As test elementary process we took a simple example: $e^+e^-$ $\rightarrow$
$( Z^0 )$ $\rightarrow$ $\nu \bar{\nu} \gamma$.
The physical conditions for the integration are taken to be  identical to
those of ref.$\lbrack 4 \rbrack$( the center of mass energy $W$ = 105
GeV and the polar angle cut for the photon $165^{\circ}$ $>$ $\theta_{\gamma}$
$>$ $15^{\circ}$), except for the energy threshold of the photon $E_{\gamma}$  $>
$ $1/W$  GeV. 
Besides this test the new algorithm was  applied to the
process $e^+e^-$ $\rightarrow$ $t \bar{t} Z^0$, where $t$ $\rightarrow$ 
 $b f \bar{f}$ and $Z^0$ $\rightarrow$ $f \bar{f}$ $\lbrack 5 \rbrack$.
In this calculation, two variables are used to distinguish the
different helicity combinations and another one to select decay modes of $Z^0$
and $W^{\pm}$, respectively, in addition to 20 variables for the phase space.
The new version  worked well even for this 23-dimensional integration and
 event generation.

By using the new version, we can integrate even a singular function and
reproduce the behavior of the function by generating events with weight one.
There are,  however, the following restrictions.
{\small\begin{description}
\item{1)} The maximim number of wild variables is limited to 15.
\item{2)} There should be no strong correlation among the wild variables. 
If there is, singular points may run continuously along a diagonal line of the
integral volume, to which our algorithm cannot be applied.
We should carefully choose the integration variables to  avoid such
correlations. 
\end{description}}
\par\noindent
In spite of these limitations, this new version of {\small BASES / SPRING}
will find many applications in the the field of very high  energy physics.