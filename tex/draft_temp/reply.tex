%#BIBTEX jbibtex intro
%\documentclass[final,5p,times,twocolumn]{elsarticle}
\documentclass[a4paper,11pt]{article}
%\usepackage{hyperref}
\usepackage{graphics}
\usepackage{amssymb}
\usepackage{amsmath}
\usepackage{latexsym}
\usepackage{supertabular}
\usepackage{slashed}
\usepackage{booktabs}
\usepackage{cases}
\usepackage{bigints}
\usepackage{mysty}
\usepackage[abs]{overpic}
%\usepackage{natbib}
%% \usepackage{lineno}

%%% global definitions %%%%
\newcommand{\nn}{\nonumber}
\newcommand{\simlt}{\lower.5ex\hbox{$\; \buildrel < \over \sim \;$}}
\newcommand{\simgt}{\lower.8ex\hbox{$\; \buildrel > \over \sim \;$}}
\newcommand{\resizeall}{\center\resizebox{0.8\textwidth}{!}}
\newcommand{\hy}{\,-\,}

%%% local definitions %%%
\newcommand{\dsol}{\Delta_{21}}
\newcommand{\ddsol}{2\Delta_{21}}
\newcommand{\drct}{\Delta_{31}}
\newcommand{\ddrct}{2\Delta_{31}}
\newcommand{\datm}{\Delta_{32}}
\newcommand{\ssol}{\sin^22\theta_{12}}
\newcommand{\srct}{\sin^22\theta_{13}}
\newcommand{\gwkty}{$\,{\rm GW_{th}}\cdot$kt$\cdot$year\,}
\newcommand{\dms}{\Delta m^2_{21}}
\newcommand{\dmr}{\Delta m^2_{31}}
\newcommand{\dcT}{\Delta\chi^2}
\newcommand{\dcTm}{(\Delta\chi^2)_{\rm min}}
\newcommand{\cT}{\chi^2}
\newcommand{\ve}{\bar{\nu}_e}
\newcommand{\exposure}{${\rm 20 \,GW_{th}}$$\cdot$5kt (12\% free-proton
weight fraction)$\cdot$5yrs}
%% if you use PostScript figures in your article
%% use the graphics package for simple commands
\usepackage{graphics}
%% or use the graphicx package for more complicated commands
%% \usepackage{graphicx}
%% or use the epsfig package if you prefer to use the old commands
%% \usepackage{epsfig}

%% The amssymb package provides various useful mathematical symbols
\usepackage{amssymb}
\usepackage{amsmath}
%% The amsthm package provides extended theorem environments
%% \usepackage{amsthm}

%% The lineno packages adds line numbers. Start line numbering with
%% \begin{linenumbers}, end it with \end{linenumbers}. Or switch it on
%% for the whole article with \linenumbers after \end{frontmatter}.
%% \usepackage{lineno}

\usepackage{latexsym}
\usepackage{supertabular}
\usepackage{slashed}
\usepackage{booktabs}
\usepackage{cases}
\usepackage{bigints}
\biboptions{sort&compress}
%% natbib.sty is loaded by default. However, natbib options can be
%% provided with \biboptions{...} command. Following options are
%% valid:

%%   round  -  round parentheses are used (default)
%%   square -  square brackets are used   [option]
%%   curly  -  curly braces are used      {option}
%%   angle  -  angle brackets are used    <option>
%%   semicolon  -  multiple citations separated by semi-colon
%%   colon  - same as semicolon, an earlier confusion
%%   comma  -  separated by comma
%%   numbers-  selects numerical citations
%%   super  -  numerical citations as superscripts
%%   sort   -  sorts multiple citations according to order in ref. list
%%   sort&compress   -  like sort, but also compresses numerical citations
%%   compress - compresses without sorting
%%
%% \biboptions{comma,round}

% \biboptions{}
%%% global definitions %%%%
\newcommand{\nn}{\nonumber}
\newcommand{\simlt}{\lower.5ex\hbox{$\; \buildrel < \over \sim \;$}}

%%% local definitions %%%
\newcommand{\dsol}{\Delta_{21}}
\newcommand{\ddsol}{2\Delta_{21}}
\newcommand{\drct}{\Delta_{31}}
\newcommand{\ddrct}{2\Delta_{31}}
\newcommand{\datm}{\Delta_{32}}
\newcommand{\ssol}{\sin^22\theta_{12}}
\newcommand{\srct}{\sin^22\theta_{13}}
\newcommand{\gwkty}{$\,{\rm GW_{th}}\cdot$kt$\cdot$year\,}
\newcommand{\dms}{\Delta m^2_{21}}
\newcommand{\dmr}{\Delta m^2_{31}}
\newcommand{\dcT}{\Delta\chi^2}
\newcommand{\dcTm}{(\Delta\chi^2)_{min}}
\newcommand{\cT}{\chi^2}
\newcommand{\ve}{\bar{\nu}_e}
\newcommand{\exposure}{${\rm 20 \,GW_{th}}$$\cdot$5kt(12\% free proton
weight fraction)$\cdot$5yrs}
%\journal{Physics Letters B}
\journal{arXiv:hep-ph}
%\journal{Preliminary Draft}

\begin{document}

%\noindent {\bf\Large  Reply to the referee report}\\

\noindent Authors: Shao-Feng Ge, Kaoru Hagiwara, Naotoshi Okamura,
Yoshitaro Takaesu

\noindent Title:  Determination of mass hierarchy with medium baseline reactor
neutrino experiments\\\\

\noindent Dear JHEP Editor,\\

\noindent We thank you for sending us your decision of our paper ``Determination
of mass hierarchy with medium baseline reactor neutrino experiments'',
together with valuable comments from the referee. We also thank the referee for giving us critical comments on the
manuscript. Here are one-by-one replies to the referee's comments.\\

\noindent {\bf Comments 1}\\
In particular, in the present note Fig 9 shows the danger (described in
Ref 20) of a naive interpretation of $\chi^2$ in terms of a probability of success. Nonetheless the idea that a 3 sigma determination of the hierarchy
(which suggests to the casual reader who hasn't  seen or understood
Fig.9 a 99.7 percent confidence in the result) is possible with a 5 kton
detector is stressed in the abstract and in the conclusions.\\

\noindent {\bf Reply 1}\\
Although we do not intend to stress the successful determination of the
mass hierarchy with the three-sigma confidence for a 5 kton experiment, it is the result of our analysis that
the mean of $\dcTm$ is $\sim 9.5\,(3.1)$ with the $(a,b)
= (2,0.5)\% \,((3,0.5)\%)$ energy resolution after 5 years of running
for our default experimental setting. Since our analysis is an optimistic one,
we interpret this result as follows: we need ``at least'' 5 (15) years of
running with the $(a,b)=(2,0.5)\% \,((3,0.5)\%)$ energy
resolution in order to ``expect $\dcTm \sim 9$ on average'' for the
experimental setting. 

We would also like to refer to the probability of
determining the right mass hierarchy, which includes fluctuations in the
data, in the Abstract and Conclusion to
avoid misunderstandings of readers.

To make these points clearer, we modified the relevant texts with the
 comments on the probability of determining the right mass hierarchy in the Abstract and Conclusion as below. (Line numbers are for the
 revised manuscript.)\\
 
\noindent [Abstract: line 7-13]\\

\noindent {\it Original}\\
for a 5 kton detector (with 12\% weight fraction of free
proton) placed at $L \sim 50$ km away from a $20 \,{\rm GW_{\rm th}}$
reactor, 3$\sigma$ determination needs 14 years of running with $a=3\%$
and $b=0.5\%$, which can be reduced to 5 years if $a=2\%$ and $b=0.5\%$.\\

\noindent {\it Revised}\\
for a 5 kton
 detector (with 12\% weight fraction of free proton) placed at $L \sim
 50$ km away
 from a $20 \,{\rm GW_{\rm th}}$ reactor, an experiment
 would determine the mass hierarchy with $\dcTm \sim 9$ on average after 5
 (15) or
 more years of running if the energy resolution
 $(a,b)=(2,0.5)\%\,((3,0.5)\%)$ is achieved. The
 probability that an experiment with the expectation of $\overline{\dcTm} = 9$ resolves the mass hierarchy is estimated to be $\sim 90\%$ by
 taking into account statistical fluctuation in the data.\\

\noindent [Discussions and Conclusion: line 10-15]\\

\noindent {\it Original}\\
$3 \sigma$ determination of the mass hierarchy is
possible for an
experiment with \exposure\, exposure if an energy
resolution of $(a,b) = (2,0.75)\%$ is achieved, while a factor of three
larger or longer experiment is needed to achieve the same goal for the
energy resolution of $(a,b)=(3,0.75)\%$.\\

\noindent {\it Revised}\\
For a 5 kton
 detector (with 12\% weight fraction of free proton) placed at $L \sim
 50$ km away
 from a $20 \,{\rm GW_{\rm th}}$ reactor, an experiment would determine the
 mass hierarchy
 with $\dcT\sim 9$ on average after five or more years of running if the
 energy resolution of
$(a,b)=(2,0.5)\%$ is achieved, while a
factor of three larger or longer experiment is needed to achieve the
same goal for the energy resolution of $(a,b)=(3,0.5)\%$.\\

\noindent [Discussions and Conclusion: line 3-4 in the last paragraph]\\

\noindent {\it Added}\\
For instance, it is estimated to
be $\sim 90\%$ for an experiment with the expectation of $\overline{\dcTm} = 9$.\\\\


\noindent {\bf Comments 2}\\
Similarly, the main limiting factor in the determination of the neutrino
mass eigenstate mass differences (and probably also in the determination
of the hierarchy) is caused by the unknown energy response of the
detector, as was stressed in Ref. 18. As these have not been considered
in the present study, the figures given for the precision to which these
mass differences can be determined would appear overly
optimistic. Indeed with the knowledge of the energy response at KamLAND,
around 2 percent, a 1 percent error in $\Delta M_{21}^2$ seems quite
plausible. $\Delta M_{31}^2$ is somewhat more robust since it relies upon a
higher energy part of the spectrum where the energy calibration is
easier.\\

\noindent {\bf Reply 2}\\
We agree that our analysis is optimistic in the sense that we did not
consider several limiting factors such as unknown non-linear responses of a detector, interference
effects among neutrino fluxes from several reactors or the finite
bin-size effect. However, the reliable estimation of those effects depend on a specific experimental conditions, and we do not discuss them in detail in this study. Instead,
 we would like to add a comment on the possible reduction of the
 sensitivity due to those factors in the Conclusion as below. \\
% Because a realistic estimation of the unknown non-linear response is
% possible after constructing and calibrating a detector, we do not
% investigate this issue in this study.\\

\noindent [Discussions and Conclusion: the 3rd paragraph]\\

\noindent {\it Added}\\
It should be noted that these estimated sensitivities can be reduced
in real experiments by factors we have not considered in this study,
such as unknown non-linear responses of a detector [18], interference
effects among neutrino fluxes from several reactors [17] or the finite
bin-size effect [15, 19]. Reliable estimations of these factors
 depend on actual experimental conditions, and they should be
assessed carefully by each experiment.\\\\

\noindent {\bf Comments 3}\\
\noindent 1) The strategy outlined in the beginning of section 5 to minimize $\chi^2$ without simulations is interesting and one of the main results of
the paper. It would be useful if the iterations, mentioned under Eq
(5.2), could be explained more explicitly. In particular, (5.2) itself appears to apply a linear approximation beyond its region of validity
for variations in $\Delta M_{31}^2$, since at low energies and or long
baselines these may shift a peak in the spectrum down to a minimum and
then up to the next peak. Are the iterations mentioned here are a way to
solve this problem and proceed step by step linearly?\\

\noindent {\bf Reply 3}\\
We iterate the process by replacing the initial parameter values  by the
fitted ones, and stop when no significant shifts are identified.

Our method can find only the local minimum nearest to the input
parameter values. Generally speaking, we should check that the found minimum is
the global minimum or not in some way. In this study we have checked
that
 the found minimum is the global minimum 
even for the wrong mass hierarchy assumption 
by trying different input values of $|\dmr|$ over its
$3\sigma$ range  since $|\dmr|$ is
the almost only relevant parameter to the $\chi^2$ minimization.

However, this result is somewhat expected from the following
reason. Thanks to the accurately measured neutrino parameters, the true $\chi^2_{\rm
min}$ for the right mass-hierarchy assumption is expected to be
the nearest local minimum, and we can find the true minimum by iterating our
$\chi^2$ minimization procedure in this case.

Even when the wrong mass-hierarchy is assumed, our method is also expected to
converge to the global minimum after a few iterations since the $\chi^2$
function does not change significantly by merely replacing the mass hierarchy
assumption; otherwise, it would be an easy task to determine the mass hierarchy.\\

% we might not have precisely measured the neutrino
% parameters almost irrelevantly to the mass hierarchy patterns.\\

\noindent For the above discussion, we added the following sentences.\\

\noindent [Sec.~5 Statistical uncertainty of the sensitivity: the 2nd
paragraph in P.14]\\

\noindent {\it Added}\\
We find that the global minimum of $\chi^2$ is reached
after a few iterations (of updating the initial parameter values by the
fitted ones) even when the wrong mass hierarchy is assumed in
the fit. This is because the $\chi^2$ function does not change drastically by
merely replacing the mass hierarchy pattern, as can be inferred from the
relatively small shifts of the fitted parameters in Fig. 4.
% is rather smooth in the allowed region of the parameter
% space including the mass hierarchy pattern. as can be inferred from the
% fitted parameter values plotted in Fig. 4.
 When applying our method to
a generic problem, absence of lower local minima should be checked by some methods. \\\\
% Similarly, the true $\chi^2_{\rm min}$
%  for the wrong mass-hierarchy assumption is expected to be the nearest
%  minimum too since the functional form of the $\chi^2_{\rm min}$ is not
% affected significantly just by changing the mass hierarchy assumption.
% % since the mass hierarchy difference in the neutrino energy spectrum is
% % small. 
% Therefore, we can find the true $\chi^2_{\rm min}$ in this case using our method.\\\\texetexetexetexetexetexetexe
% Even if the nearest minimum of the input parameter values is not the
% true minimum by any possibility, we can find it by
% trying several sets of input values within the uncertainties of the
% parameters and comparing the resulted minima. \\


\noindent {\bf Comments 4}\\
\noindent 2) On page 14 the authors write ``It is plausible to assume that
$\chi^2$ min corresponding to the right mass-hierarchy determination
follows the normal distribution.'' This assumption is essential to what
follows. How can it be justified?\\

\noindent {\bf Reply 4}\\
\noindent Although this is an assumption, it is a reasonable one. If we
assume high enough statistics such that the event number in each bin is expected to follow a Gaussian distribution, it is natural to think that
$\dcTm$ approximately follows a normal distribution. To
explain this point more clearly, we modified the relevant sentences as follows. \\

\noindent [Sec.~5 Statistical uncertainty of the sensitivity: the last
paragraph in P.14]\\

\noindent {\it Original} \\
It is plausible to assume that $\dcTm$ corresponding to the right
mass-hierarchy determination follows the normal distribution
with the mean $\overline{\dcTm}$
 and the standard deviation $\delta
\left\{ \dcTm \right\}$ [20]. \\

\noindent {\it Revised}\\
 It is plausible to assume that $\dcTm$ corresponding to the right
mass-hierarchy determination follows the normal distribution
with the mean $\overline{\dcTm}$
 and the standard deviation $\delta
\left\{ \dcTm \right\}$ from the following
reason (cf. ref. [20]). First, we can assume that the
number of events in the $i_{\rm th}$ bin
      follows the normal
distribution with the mean $\overline{N_i}$ and the standard
      deviation
      $\sqrt{\overline{N}_i}$ in the limit of high
      statistics\footnote{In order to reduce the errors due to finite
      statistics, our program combines adjacent bins whenever the
      expected event number is smaller than ten. This may partly explain
 the good agreements between our simulation results and the expectation
 curve in Fig. 9.}. 
Then, expressing the
observed event number in the $i_{\rm th}$ bin as $N^{\rm data}_i =
\overline{N}_i +\Delta N_i$ and neglecting the second term of eq.~(5.1),
 the $\dcTm$ is written as
\begin{align*}
&\dcTm = \chi^2_{\rm min}({\rm wrong\,MH}) -\chi^2_{\rm min}({\rm
 right\,MH})\nn\\
&\simeq \sum_{i=1}^{\rm nbins}\frac{\left\{N^{\rm data}_i -N^{\rm
 fit({\rm wrong\, MH})}_i\right\}^2}{N^{\rm data}_i} -\sum_{i=1}^{\rm
 nbins}\frac{\left\{N^{\rm data}_i -N^{\rm
 fit({\rm right \,MH})}_i\right\}^2}{N^{\rm data}_i}\nn\\
&\simeq \sum_{i=1}^{\rm nbins}\left[\frac{\left\{\overline{N}_i+\Delta N_i -N^{\rm fit({\rm
 wrong\,MH})}_i\right\}^2}{N^{\rm data}_i}  
 -\frac{(\Delta N_i)^2}{N^{\rm data}_i} \right]\nn\\
% +{\cal O}\left( \frac{\left|\overline{N}_i -N^{\rm
%  fit(right\,MH)}_i\right|}{\sqrt{N^{\rm data}_i}} \right) \right]\nn\\
 &\simeq \sum_{i=1}^{\rm nbins}\left[\frac{\left\{\overline{N}_i -N^{\rm
 fit({\rm wrong\,MH})}_i\right\}^2}{\overline{N}_i}\right.\nn\\ 
&\left.\hspace{4em}+\left\{ 2\frac{\overline{N}_i-N^{\rm fit({\rm
 wrong\,MH})}_i}{\overline{N}_i} -\frac{\left\{\overline{N}_i -N^{\rm
 fit({\rm wrong\,MH})}_i\right\}^2}{\overline{N}_i^2} \right\}\Delta N_i \right]\nn\\
%  &\hspace{4em}\left. +{\cal O}\left( \frac{ \left| \overline{N}_i -N^{
%  \rm fit( right\,MH ) }_i \right| }{ \sqrt{ \overline{N}_i } } \right) \right].
\label{eq:dcTmnorm}
\end{align*}
Here we set $N^{{\rm fit}({\rm right\,MH})}_i \sim \overline{N}_i$, naturally requiring that the
theoretical prediction with the right mass hierarchy fits the data well
in order to declare the determination of the mass hierarchy. 
In the last line, we neglect ${\cal O}((\Delta N_i)^2/\overline{N}_i^2)$ terms.
% The order $\left| \overline{N}_i -N^{\rm fit
%       (right\,MH)}_i\right|/\sqrt{\overline{N}_i}$ term represents the
%       goodness of fit of the theoretical prediction to the data;
%       if $\chi^2_{\rm min}({\rm right\,MH})/{\rm d.o.f.} \sim 1$, the term is expected to be
%       nearly zero. 
Since $\Delta N_i$ follows the normal distribution, 
the $\dcTm$ also approximately follows a normal distribution if the following conditions are satisfied:
\begin{enumerate}
\item The statistics is high enough to assume the Gaussian
      distribution of the event number in each bin.
 \item The statistical part, the first term in eq.~(5.1),
dominates the $\chi^2$ function.
\item The theoretical prediction with the
right mass hierarchy assumption fits the data well so that
      $N^{{\rm fit}({\rm right\,MH})}_i \sim \overline{N}_i$.
% the order $\left| \overline{N}_i -N^{\rm fit
%       (right\,MH)}\right|/\sqrt{\overline{N}_i}$ term can be neglected in
%       eq.~(5.11).
\end{enumerate}
The mean and the standard deviation of the normal distribution should
then be those obtained with our method,
$\overline{\dcTm}$ (eq.~(5.9)) and $\delta\left\{ \dcTm
\right\}$(eq.~(5.8)), respectively.\\\\
%
% \begin{enumerate}
% \item We can assume that the number of events in $i_{\rm th}$ bin follows the normal
% distribution with the mean of $\overline{N_i}$ and the standard deviation of
%       $\sqrt{\overline{N}_i}$ in the limit of high statistics. 
% \item The contribution of each bin to the $\dcT$ follows a normal
% distribution.\\
% This can be understood as follows. First, we express the observed event
%       number in the $i_{\rm th}$ bin as $N^{\rm data}_i = \overline{N}_i
%       +\Delta N_i$.
% The
%       $\dcT$ is then written as
% \begin{align}
% \dcTm &= \chi^2_{\rm min}({\rm wrong\,MH}) -\chi^2_{\rm min}({\rm right\,MH})\nn\\
% &\simeq \sum_{i=1}^{\rm nbins}\frac{\left\{N^{\rm data}_i -N^{\rm
%  fit({\rm wrong\, MH})}_i\right\}^2}{N^{\rm data}_i} -\sum_{i=1}^{\rm nbins}\frac{\left\{N^{\rm data}_i -N^{\rm
%  fit({\rm right \,MH})}_i\right\}^2}{N^{\rm data}_i}\nn\\
% \end{align}
% Assuming that the theoretical prediction with the right mass hierarchy assumption fits
%       the data well, we set $\overline{N}_i \simeq N^{\rm fit(right
%       \,MH)}_i$ and obtain
% \begin{align}
% \dcTm &\simeq \sum_{i=1}^{\rm
%  nbins}\left[\frac{\left\{\overline{N}_i+\Delta N_i -N^{\rm
%  fit({\rm wrong\,MH})}_i\right\}^2}{N^{\rm data}_i}
%  -\frac{(\Delta
%  N_i)^2}{N^{\rm data}_i}\right]\nn\\
%   &\simeq \sum_{i=1}^{\rm nbins}\left[\frac{\left\{\overline{N}_i-N^{\rm fit({\rm
%  wrong\,MH})}_i\right\}^2}{\overline{N}_i}
%  +2\frac{\overline{N}_i-N^{\rm fit({\rm wrong\,MH})}_i}{\overline{N}_i}\Delta N_i\right].
% \end{align}
% In the last line, we ignored $\sqrt{\overline{N}_i}$ comparing to
%       $\overline{N}_i$, assuming a large statistics.
% Since $\Delta N_i$ follows the normal distribution, the contribution to
%       $\dcTm$ from each bin
%       also follows a normal distribution.\\
% \item Therefore, $\dcTm$ approximately follows a normal
%       distribution if the statistical part, eq.~(3.4), dominates the
%       $chi^2$ function. The mean and the standard deviation of the
%       distribution are
%       those estimated by our method, $\overline{\dcTm}$ and $\delta
%       \dcTm$, respectively.
% \item In order
%       to reduce errors due to finite statistics, our program combines
%       adjascent bins whenever the expected event number is smaller than
%       10. This should lead to non-Gaussian behavior for low statistics
%       experiments, as may be indicated by the slightly lower probability
%       observed for simulation results in the low $\dcTm$ cases as
%       compared to the theoretical (solid) curve in Fig. 9.
% \end{enumerate}
% %\noindent This conclusion is also supported by the actual simulation
% %in Fig. 2 in Ref.~20.\\\\ 

\noindent {\bf Comments 5}\\
1) The neutrino flux per GW kT year in this study appears to be 60-100
percent higher than in the 3 other studies mentioned. 20 percent of this
may be due to the fact that the detector includes 12 percent instead of
10 percent free protons. This study derives the flux from first
principles, whereas the others scale the observed fluxes at other such
experiments such as Daya Bay. Considering that this study is the only
one to claim that a 5 kton detector may be big enough, the total flux
normalization observed is an important point. By scaling Daya Bay flux,
other studies automatically consider effects such as downtime for the
reactors and dead time for the detectors. Is this the reason for the
discrepancy between the flux at Daya Bay (scaled explicitly for example
in Ref. 18) and that presented here?\\

\noindent Also the current article appears to have significantly more neutrino
events per GW kton year than studies which have scaled the fluxes
observed at other experiments (for example, only 5 times less flux than
Ref. [18] despite having half as many reactors and 1/4 as much target
mass.) \\

\noindent {\bf Reply 5}\\
\noindent  As the referee pointed out, we did not take into account the
possible downtime of reactors and deadtime of a detector. However, we
suspect that the referee's concern on the possible 60-100\% excess in
our flux comes from a different flux estimation from
ours. One reason is that our
flux estimated from the first principle is only 3\% larger than that
used in Ref. [18]. Another reason is the following.  Although the referee seemed to point out that our event
number should be smaller than that in Ref. [18] by about a factor of
eight, we can see roughly that the number of events observed in our default
setting is about five times smaller than that in the reference as follows:
\begin{align}
 \frac{20 [{\rm GW_{th}}]}{40 [{\rm GW_{th}}]}\frac{5 [{\rm kton}]}{20
 [{\rm kton}]}\frac{1/50^2 [1/{\rm km^2}]}{1/60^2 [1/{\rm km^2}]}\times
 1.03
\,({\rm flux}) \sim 1/5.4.
\end{align}
Therefore, we think that our flux is not much higher than those in the
references as the referee concerned.

We have also checked that our results do not change significantly even if we normalize our
flux with the Daya Bay flux used in Ref. [18]. 

Considering the above discussion, we added the following sentences for clearer comparison with the references.\\

\noindent [Sec.~5 Statistical uncertainty of the sensitivity: the 4th
paragraph
in P.17]\\

\noindent {\it Added}\\
Although we do not take into account the
possible downtime of reactors and deadtime of a detector, the
reactor neutrino flux (2.3) calculated from the first
principle in this study
is only 3\% larger than that used in ref.~[18], whose
analysis
includes these factors by using the measured flux by the Daya Bay
experiment. We
have checked that our results do not change significantly
even if we normalize our flux with the Daya Bay flux.\\\\
% We can see roughly that the number of events observed in our default
% setting is five times smaller than that in the reference as follows:
% \begin{align}
%  \frac{20 [{\rm GW_{th}}]}{40 [{\rm GW_{th}}]}\frac{5 [{\rm kton}]}{20
%  [{\rm kton}]}\frac{1/50^2 [1/{\rm km^2}]}{1/60^2 [1/{\rm km^2}]}\times
%  1.03
% \,({\rm flux}) \sim 1/5.4.
% \end{align}\\\\


\noindent {\bf Comments 6}\\
2) In Refs 14 and 17 the probabilities of success are simply the
fraction of the simulations in which the Fourier analysis yields the
correct hierarchy. As these references did not attempt $\chi^2$ analyses,
one does not expect an exact match. However in Ref 18 the same data was
analyzed using both methods and the resulting chance of successes were
comparable. As the Fourier analysis observables (RL, PV, etc) are
insensitive to the overall scale, this method is insensitive to $\Delta
M^2_{31}$ and as the authors note, fitting the other parameters does
not strongly affect $\chi^2$, so it seems difficult to see how the 83
percent vs 90 percent descrepency could be due to marginalization as is
suggested here.

3a) The simultaneous comparisons to Refs 14 and 17 are important but a
bit difficult to follow. In particular, it appears that the results are
consistent with Ref 17 (93.4 vs 93.6 percent) but not with Ref 14. Is
this the correct interpretation? It could be useful for the reader if
this were written explicitly.

3b) In this case however, the good agreement with Ref 17 suggests that
there is no difference in the interpretation of the probability of
success (which is identical in Refs 14 and 17 except for the
determination of the threshold value of RL+PV for the hierarchy
determination.)\\

\noindent {\bf Reply 6}\\
The main cause of the difference of the $83\%$ vs. 90\% discrepancy is
the Gauss error function ${\rm erf}(\sqrt{\dcTm}/\sqrt{2})$ we included
in estimating the probability (5.12). If we set ${\rm
erf}(\sqrt{\dcTm}/\sqrt{2}) = 1$, our probability becomes $88.4\%$ and
$95.6\%$ for the settings of ref. [14] and [17], respectively, showing
the rough agreement with the probabilities quoted in those references,
$\sim 90\%$ and 93.4\%, respectively. 
%
% This
% function corresponds to the sensitivity of each experiment
% for determining the right mass hierarchy.  Since this is always less
% than unity, our probability is basically smaller than that of Refs. [14,
% 17], where the
% authors just counted successful pseudo-experiments in their probability estimation, without considering
% the sensitivity of those experiments.
%
% Our probability
% becomes significantly smaller than that of Refs. [14, 17] when the mean $\dcTm$
% is not large. This is the case of the 83\% vs. 90\% discrepancy, where
% $\overline{\dcTm} \sim 5.7$. 
%
% On the other hand, when $\overline{\dcTm}$
% is large enough, our probability and that of Ref. [14] and [17] are close to
% each other since the effect of the function ${\rm
% erf}(\sqrt{\dcTm}/\sqrt{2})$ becomes negligible in
% that case. This is the case of the comparison with Ref. [17], 93.6\%
% vs. 93.4\%, where $\overline{\dcTm} \sim 11.7$.
%
% We modified the corresponding text to clearly state that the Gauss error function
% mainly causes the difference between our probability and that of Ref. [14]
% as below.\\
We modified the corresponding texts as below.\\

\noindent [Sec.~5 Statistical uncertainty of the sensitivity: the 2nd last
paragraph of the section]\\

\noindent {\it Original}\\
Somewhat smaller probability of our estimate
$\sim 83\%$ may reflect the
factor ${\rm erf}(\sqrt{x}/\sqrt{2})<1$ in eq.~(5.12), whose
effect can be significant when $\dcTm$ is not large. \\
% Another possible reason is
% that only one set of parameter values was studied in their analysis
% without marginalizing the
% probabilities as pointed out in ref. [18]. \\

\noindent {\it Revised} \\
Somewhat smaller probability of our estimate
$\sim 83\%$ mainly reflects the
factor ${\rm erf}(\sqrt{x}/\sqrt{2})<1$ in eq.~(5.12),  whose
effect can be significant when the expected $\overline{\dcTm}$ is not
large.  If we set ${\rm
erf}(\sqrt{\dcTm}/\sqrt{2}) = 1$, our probability becomes $88.4\%$.
On the other hand, this effect becomes small for a large
$\overline{\dcTm}$ case
 as can be seen in the
rough agreement between the probability of Ref. [17] (93.4\%) and our probability ($\sim 93.7\%$), where
$\overline{\dcTm} \sim 11.7$. In this case, we obtain the $95.6\%$
probability if we set ${\rm
erf}(\sqrt{\dcTm}/\sqrt{2}) = 1$.\\\\


\noindent We believe that this revisions will make our paper more useful
to the readers.\\
\noindent Thank you very much.\\\\\\
\noindent Sincerely yours,\\

\noindent Yoshitaro Takaesu, for the authors
 
\bibliographystyle{bib/model1a-num-names}
\bibliography{bib/neutrino,bib/neutrino_c}
\end{document}

% LocalWords:  descrepency infered cf adjascent
