\documentclass[final,5p,times,twocolumn]{elsarticle}
%% if you use PostScript figures in your article
%% use the graphics package for simple commands
\usepackage{graphics}
%% or use the graphicx package for more complicated commands
%% \usepackage{graphicx}
%% or use the epsfig package if you prefer to use the old commands
%% \usepackage{epsfig}

%% The amssymb package provides various useful mathematical symbols
\usepackage{amssymb}
\usepackage{amsmath}
%% The amsthm package provides extended theorem environments
%% \usepackage{amsthm}

%% The lineno packages adds line numbers. Start line numbering with
%% \begin{linenumbers}, end it with \end{linenumbers}. Or switch it on
%% for the whole article with \linenumbers after \end{frontmatter}.
%% \usepackage{lineno}

\usepackage{latexsym}
\usepackage{supertabular}
\usepackage{slashed}
\usepackage{booktabs}
\usepackage{cases}
\usepackage{bigints}
\biboptions{sort&compress}
%% natbib.sty is loaded by default. However, natbib options can be
%% provided with \biboptions{...} command. Following options are
%% valid:

%%   round  -  round parentheses are used (default)
%%   square -  square brackets are used   [option]
%%   curly  -  curly braces are used      {option}
%%   angle  -  angle brackets are used    <option>
%%   semicolon  -  multiple citations separated by semi-colon
%%   colon  - same as semicolon, an earlier confusion
%%   comma  -  separated by comma
%%   numbers-  selects numerical citations
%%   super  -  numerical citations as superscripts
%%   sort   -  sorts multiple citations according to order in ref. list
%%   sort&compress   -  like sort, but also compresses numerical citations
%%   compress - compresses without sorting
%%
%% \biboptions{comma,round}

% \biboptions{}
%%% global definitions %%%%
\newcommand{\nn}{\nonumber}
\newcommand{\simlt}{\lower.5ex\hbox{$\; \buildrel < \over \sim \;$}}

%%% local definitions %%%
\newcommand{\dsol}{\Delta_{21}}
\newcommand{\ddsol}{2\Delta_{21}}
\newcommand{\drct}{\Delta_{31}}
\newcommand{\ddrct}{2\Delta_{31}}
\newcommand{\datm}{\Delta_{32}}
\newcommand{\ssol}{\sin^22\theta_{12}}
\newcommand{\srct}{\sin^22\theta_{13}}
\newcommand{\gwkty}{$\,{\rm GW_{th}}\cdot$kt$\cdot$year\,}
\newcommand{\dms}{\Delta m^2_{21}}
\newcommand{\dmr}{\Delta m^2_{31}}
\newcommand{\dcT}{\Delta\chi^2}
\newcommand{\dcTm}{(\Delta\chi^2)_{min}}
\newcommand{\cT}{\chi^2}
\newcommand{\ve}{\bar{\nu}_e}
\newcommand{\exposure}{${\rm 20 \,GW_{th}}$$\cdot$5kt(12\% free proton
weight fraction)$\cdot$5yrs}
%\journal{Physics Letters B}
\journal{arXiv:hep-ph}
%\journal{Preliminary Draft}

\begin{document}

\section{Discussions and Conclusion}
\label{conclusion}
In this paper we have investigated the sensitivity of 
medium baseline reactor electron-antineutrino oscillation experiments for
determining the neutrino mass hierarchy by performing the standard
$\cT$ analysis.
We carefully study the impacts of the
systematic uncertainty of the energy resolution 
$(\delta E/E)^2 = \left(a/\sqrt{E/{\rm MeV}}\right)^2 +b^2$
and find that the optimal baseline length and the sensitivity
strongly depend on
the energy resolution.
% We confirm the previous
%findings~\cite{Learned:2006wy,Batygov:2008ku,Zhan:2009rs,Ghoshal:2010wt,Ghoshal:2012ju,Ciuffoli:2012iz,Qian:2012xh} that t
 The optimal baseline length
which maximizes the mass hierarchy resolving power of the experiment is
found to depend slightly on the energy resolution, preferring the length
slightly shorter than 50 km for the energy resolution of
$(a,b)=(3,0.75)\%, (3,1)\%, (2,0.75)\%$ and $(2,1)\%$.
At the optimal baseline length,
the energy resolution better than the
$3\%/\sqrt{E/{\rm MeV}}$ level is needed to determine the neutrino mass
hierarchy pattern. 
Three-sigma determination of the mass hierarchy is possible for an
experiment with \exposure\, exposure if an energy
resolution of $(a,b) = (2,0.75)\%$ is achieved, while a factor of 3
larger or longer experiment is needed to achieve the same goal for the
energy resolution of $(a,b)=(3,0.75)\%$.  
% The optimal baseline length
% $L_{opt}$ is obtained at around 50 km for the energy resolution of $dE_{vis}/E_{vis} = (2-3)/\sqrt{E_{vis}}$\%, but it    
% shows the strong dependence on the energy resolution, especially on the
% statistical uncertainty $a$ of the resolution; $L_{opt}$ is
% shifted to $\sim 48, 40$ and $30$ km for $a = 4, 5$ and 6\%, respectively.
%  The shorter baseline length $L<50$ km results in small $\dcT$ since the
%  difference between
% the mass hierarchy is absorved by the $|\dmr|$ shift, while the longer
% baseline length suffers from the small statistics. Inclusion of the
% systematic uncertainty in the energy resolution affects not only on the
% $L_{opt}$ but also on the significance for the mass hierarchy determination
% itself. It reduces the $\dcT_{min}$ by a factor of 1.5 for
% $b=1\%$ w.r.t. $b=0\%$. However, even in the existence of the non-zero $b$, it is found
% that the significance for $a = 2\%$ resolution is almost twice higher
% than that for $a = 3\%$ with $b=0\%$. Improvement of the statistical
% uncertainty part to $2/\sqrt{E_{vis}}\,\%$ level is therefore very important. The 3-sigma significance is
% achieved after 4 and 13 years running of the experiment for $a = 2$ and
% $3\%$ if the systematic uncertainty is reduced to $b=0.5\%$.
It is also found that this experiment can measure the neutrino
parameters, $\ssol$, $\dms$ and $|\dmr|$, very accurately as shown in (\ref{eq:param_errors}) for an
experiment of \exposure\, at
$L\sim 50$ km.
\bibliographystyle{bib/model1a-num-names}
\bibliography{bib/neutrino,bib/neutrino_c}
\end{document}
