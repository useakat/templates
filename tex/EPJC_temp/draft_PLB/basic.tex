%#BIBTEX jbibtex basic
\documentclass[final,5p,times,twocolumn]{elsarticle}
%% if you use PostScript figures in your article
%% use the graphics package for simple commands
\usepackage{graphics}
%% or use the graphicx package for more complicated commands
%% \usepackage{graphicx}
%% or use the epsfig package if you prefer to use the old commands
%% \usepackage{epsfig}

%% The amssymb package provides various useful mathematical symbols
\usepackage{amssymb}
\usepackage{amsmath}
%% The amsthm package provides extended theorem environments
%% \usepackage{amsthm}

%% The lineno packages adds line numbers. Start line numbering with
%% \begin{linenumbers}, end it with \end{linenumbers}. Or switch it on
%% for the whole article with \linenumbers after \end{frontmatter}.
%% \usepackage{lineno}

\usepackage{latexsym}
\usepackage{supertabular}
\usepackage{slashed}
\usepackage{booktabs}
\usepackage{cases}
\usepackage{bigints}
\biboptions{sort&compress}
%% natbib.sty is loaded by default. However, natbib options can be
%% provided with \biboptions{...} command. Following options are
%% valid:

%%   round  -  round parentheses are used (default)
%%   square -  square brackets are used   [option]
%%   curly  -  curly braces are used      {option}
%%   angle  -  angle brackets are used    <option>
%%   semicolon  -  multiple citations separated by semi-colon
%%   colon  - same as semicolon, an earlier confusion
%%   comma  -  separated by comma
%%   numbers-  selects numerical citations
%%   super  -  numerical citations as superscripts
%%   sort   -  sorts multiple citations according to order in ref. list
%%   sort&compress   -  like sort, but also compresses numerical citations
%%   compress - compresses without sorting
%%
%% \biboptions{comma,round}

% \biboptions{}
%%% global definitions %%%%
\newcommand{\nn}{\nonumber}
\newcommand{\simlt}{\lower.5ex\hbox{$\; \buildrel < \over \sim \;$}}

%%% local definitions %%%
\newcommand{\dsol}{\Delta_{21}}
\newcommand{\ddsol}{2\Delta_{21}}
\newcommand{\drct}{\Delta_{31}}
\newcommand{\ddrct}{2\Delta_{31}}
\newcommand{\datm}{\Delta_{32}}
\newcommand{\ssol}{\sin^22\theta_{12}}
\newcommand{\srct}{\sin^22\theta_{13}}
\newcommand{\gwkty}{$\,{\rm GW_{th}}\cdot$kt$\cdot$year\,}
\newcommand{\dms}{\Delta m^2_{21}}
\newcommand{\dmr}{\Delta m^2_{31}}
\newcommand{\dcT}{\Delta\chi^2}
\newcommand{\dcTm}{(\Delta\chi^2)_{min}}
\newcommand{\cT}{\chi^2}
\newcommand{\ve}{\bar{\nu}_e}
\newcommand{\exposure}{${\rm 20 \,GW_{th}}$$\cdot$5kt(12\% free proton
weight fraction)$\cdot$5yrs}
%\journal{Physics Letters B}
\journal{arXiv:hep-ph}
%\journal{Preliminary Draft}

\begin{document}

\section{Reactor antineutrino flux}
\label{basic}
In this section, we briefly discuss the evaluation of
how many electron antineutrinos $\bar{\nu}_e$ would be detected at
a far detector with a medium 
baseline length from a reactor. 

In a nuclear reactor, antineutrinos are
mainly produced
via beta decay of the fission products of the four radio-active isotopes,
${}^{235}U, {}^{238}U, {}^{239}P_u$ and ${}^{241}P_u$, in the fuel\footnote{Precisely speaking, there are
contributions from other isotopes such as 
${}^{240} P_u$ and ${}^{242}P_u,$ but their contributions are of the order of
0.1\% or less~\cite{Bemporad:2001qy}.}. 
The number of antineutrinos produced per fission depends on their energy
$E_{\nu}$~\cite{Vogel:1989iv}
%\footnote{There are several studies for the
%coefficients in the exponential with different
%treatments of input data and uncertainties~\cite{flux}. The difference between those results should be considered as
%the uncertainty of the neutrino flux, but we do not consider it
%explicitly in this study, including it in the overall uncertainty, see
%Sec.\ref{chi2} for more detail.}
\begin{align}
 \phi(E_{\nu}) &= \,f_{\,{}^{235}U}\exp\left(0.870-0.160E_{\nu}-0.091E_{\nu}^2\right)
 \nn\\
 &+f_{\,{}^{239}P_u}\exp\left( 0.896 -0.239E_{\nu} -0.0981E_{\nu}^2 \right) \nn\\
 &+f_{\,{}^{238}U}\exp\left( 0.976 -0.162E_{\nu} -0.0790E_{\nu}^2 \right) \nn\\
 &+f_{\,{}^{241}P_u}\exp \left( 0.793 -0.080E_{\nu} -0.1085E_{\nu}^2
 \right),
\end{align}
where $f_k$ denotes the relative fission contribution of the
isotope $k$ in a reactor fuel, derived from the fission rate $N_k^{fiss}$ ($s^{-1}$) of
isotope $k$ as
\begin{align}
f_k \equiv \frac{N_k^{fiss}}{\sum_i N_i^{fiss}}.
\end{align}
Although $f_k$ varies over time as the fuel is burned, it can be
approximated for this type of experiments with the average value of the relative
fission contributions: $f_{\,{}^{235}U}=0.58,
\,f_{\,{}^{239}P_u}=0.30, \,f_{\,{}^{238}U}=0.07$ and
$\,f_{\,{}^{241}P_u}=0.05$~\cite{Zhan:2008id}.
%the total antineutrino
%distribution over a long time period
%is approximately obtained with the averaged values of the relative
%fission contributions~\cite{Zhan:2008id}: $f_{\,{}^{235}U}=0.58,
%\,f_{\,{}^{239}P_u}=0.30, \,f_{\,{}^{238}U}=0.07$ and
%$\,f_{\,{}^{241}P_u}=0.05$.
%this type of reactor antineutrino experiment uses the accumulated number
%of antineutrinos over a long period of time.  
The event rate of antineutrinos with energy $E_{\nu}$ (MeV) at
a reactor of $P \,(\,{\rm GW_{th}})$ thermal power is then
expressed as
\begin{align}
 \frac{dN}{dE_{\nu}} = \frac{P}{\sum_k f_k \epsilon_k}\phi(E_{\nu}) \times 6.24\times 10^{21},
% f_i\phi_i(E_{\nu}) \times 6.24\times 10^{21},
\end{align}where $\epsilon_k$ is the released energy per fission of the
isotope $k$:
$\epsilon_{\,{}^{235}U}=201.7~\mbox{MeV}, \epsilon_{\,{}^{239}P_u}=210.0~\mbox{MeV}, \epsilon_{\,{}^{238}U}=205.0~\mbox{MeV}$
and $\epsilon_{\,{}^{241}P_u}=212.4$ MeV~\cite{Huber:2004xh}. The numerical factor comes 
from unit conversion, 1 GW/MeV $=6.24\times 10^{21}$.

This rate is then modulated by oscillation. The $\bar{\nu}_e$ survival probability is expressed
as
%\footnote{We neglect the earth matter effects here because such
%effects are not relevant for the energy of reactor antineutrinos and the
%baseline length studied in this paper.}
\begin{align}
 P_{ee} &= \left|\sum_{i=1}^3 U_{ei} \exp\left(-i \frac{m_i^2}{2E_i} \right)
 U^*_{ei} \right|^2 \nn\\
&= 1 -\cos^4\theta_{13}\ssol\sin^2\left(\dsol\right) \nn\\
&\hspace{1.7em} -\cos^2\theta_{12}\srct\sin^2\left(\drct\right) \nn\\
&\hspace{1.7em} -\sin^2\theta_{12}\srct\sin^2\left(\datm\right),
\label{pee1}
\end{align}
where $U_{ei}$ is the neutrino mixing matrix element
relating the electron neutrino to the mass eigenstate ${\nu_i}$. The variables $m_i$ and
$E_i$ are the mass and energy of the corresponding mass eigenstate, 
while $\theta_{ij}$ represent the neutrino mixing angles. The oscillation
phase $\Delta_{ij}$ are defined as,
\begin{equation}
 \Delta_{ij} \equiv \frac{\Delta m^2_{ij}L}{4E_{\nu}}, 
 \hspace{1em}(\Delta m^2_{ij} \equiv m^2_i -m^2_j),
\label{deltaij}
\end{equation}
with a baseline length $L$.  
%The contribution of the matter effect is
%estimated as $\sim 2
%\times 10^{-6} {\rm eV}^2 \frac{L}{2E_{\nu}}}$, which can be safely neglected, being
%significantly smaller than the smaller mass squared difference
%contribution of $\sim 7.5 \times 10^{-5} {\rm
% eV}^2$\cite{Hagiwara:2011kw}. 
We have neglected the matter effect because it is
small enough for the energy range and
 the baseline lengths we concern in this study~\cite{Hagiwara:2011kw}. In
 obtaining the second line of (\ref{pee1}) we have also ignored the tiny energy
 difference between the three mass eigenstates, $E_{\nu}\sim
 E_1\sim E_2\sim E_3$. 

To make the effect of mass hierarchy more clear, we would like to rewrite 
eq.~(\ref{pee1}) as,
\begin{align}
P_{ee} = 1 &-\cos^4\theta_{13}\ssol\sin^2\left(\dsol\right) \nn\\
& -\srct\sin^2\left(|\drct|\right) \nn\\
  &
 -\sin^2\theta_{12}\srct\sin^2\left(\dsol\right)\cos\left(2|\drct|\right) \nn\\
& \pm \frac{\sin^2\theta_{12}}{2}\srct\sin\left(\ddsol\right)\sin\left(2|\drct|\right),
\label{pee2}
\end{align}
where only the last term depends on the mass hierarchy, which takes the
plus and minus sign, respectively, for normal (NH) and inverted
hierarchy (IH),
\begin{subequations} %17:47$B$N<072(B
\begin{numcases}{\Delta m_{31}^2 \equiv }
m_3^2 -m_1^2 > 0 \hspace{1em}({\rm NH}) \\
m_3^2 -m_1^2 < 0  \hspace{1em}({\rm IH}).
\end{numcases}
\end{subequations}
 It is clear from eq.~(\ref{pee2}) that the survival probability is most
  sensitive to the mass hierarchy when $|\sin(\ddsol)| = 1$, or equivalently
\begin{subequations}
\begin{equation}
\ddsol = (2n-1)\frac{\pi}{2} \hspace{1em} (n = 1,2,3,\cdots), 
\label{condition1}
\end{equation}
and has no sensitivity at
\begin{equation}
\ddsol = n\pi \hspace{1em} (n = 0,1,2,3,\cdots),
\label{condition2}
\end{equation}
\end{subequations}
%
where $\sin(\ddsol) = 0$.
For example, at $L=50$ km, the condition (\ref{condition1}) for $n=1$ and 2 is satisfied at $E_{\nu} \sim 6$ MeV
and 2 MeV, respectively. 
%Although there is no sensitivity at
%$E_{\nu}\sim 3$ MeV, at which $\ddsol \sim \pi$ in
%eq.~(\ref{condition2}), 
The last term in eq.~(\ref{pee2}) contributes with the opposite sign at these
first and second maxima. In between, it vanishes and changes its sign at $E_{\nu} =
3~\mbox{GeV}$, 
corresponding to $n = 1$ in (\ref{condition2}). 
It is this sign change that plays an important role for the mass
hierarchy determination, which will be further discussed in the next section.  

Similar as the current reactor experiments, such as Daya Bay~\cite{DayaBay},
RENO~\cite{Ahn:2012nd} and Double Chooz~\cite{Abe:2011fz},
future medium baseline reactor antineutrino experiments can also
use free protons as targets to detect electron antineutrinos via the inverse neutron
beta decay (IBD) process, %$\bar{\nu}_e +p \rightarrow e^+ +n$.
\begin{align}
 \bar{\nu}_e +p \rightarrow e^+ +n,
\label{IBD}
\end{align}
 where $p$ and $n$ are the proton and the neutron, respectively. The threshold neutrino energy of this process is $E_{thr}\sim
m_n -m_p +m_e$, and
the cross section is~\cite{Vogel:1999zy},
 \begin{align}
  \sigma_{IBD} = 0.0952\left( \frac{E_e p_e}{\rm 1MeV^2} \right)\times
  10^{-42} \,{\rm cm^2},
%E_e \sim& E_{\nu} -m_n +m_p, \nn\\
%p_e =& \sqrt{ E_e^2 -m_e^2},
 \label{xsec}
 \end{align}
where $E_e$ and $p_e$ are the energy and momentum of the
positron, neglecting the kinetic energy of the proton and the
neutron for a MeV scale antineutrino.  The positron's energy is roughly
$E_e \sim E_{\nu} -(m_n -m_p)$.
%The
%  $\ve$ threshold energy ($E_{thr}$) for this process is $\sim 1.81 {\rm MeV}$,
%\begin{align}
% E_{thr} \sim& \,m_n -m_p +m_e \sim \,1.81 \,{\rm MeV},
%\label{Ethr}
%\end{align}
% and the produced positron energy is related to $E_{\nu}$ as
%The positron energy is related to $E_{\nu}$ as
%\begin{align}
% E_e \sim (E_{\nu} -1.3) \,\,{\rm MeV}.
% E_e \sim E_{\nu} -E_{thr} +m_e
%\label{Ee}
%\end{align}
%with the $\ve$ threshold energy of IBD, $E_{thr} \sim 1.81 {\rm MeV}$,
%and the positron mass, $m_e$. 

The produced positron then interacts with scintillator, converting its
kinetic energy to photons. Eventually, the positron annihilates with an
electron in the detector and emits two 0.5 MeV photons. The total
energy of those photons are then accumulated as the visible energy,
$E_{vis}$, which is a sum of the positron's total and one electron's rest energies,
\begin{align}
 E_{vis} \sim& E_e +m_e 
 \sim ( E_{\nu} -0.8 ) \,{\rm MeV} .
\end{align} 

Finite energy resolution of the detector then distorts the true visible
energy, $E_{vis}$, to the finally observed one, $E^{obs}_{vis}$. This effect can be modeled by a detector response function $G(E_{vis}
-E^{obs}_{vis},\delta E_{vis})$ with the energy resolution $\delta E_{vis}$. 
In this study, we take the normalized
gaussian function as the response function, i.e.,
\begin{align}
 G(E_{vis}-E^{obs}_{vis},\delta E_{vis}) = \frac{1}{\sqrt{2\pi}\delta
 E_{vis}}\exp\left\{ -\frac{ \left(E_{vis} -E^{obs}_{vis}\right)^2  }{ 2(\delta E_{vis})^2
 } \right\}.
\quad
\label{eq:G}
\end{align} 
The detector energy resolution~\cite{Zhan:2009rs}
 \begin{align}
  \frac{ \delta E_{vis} }{E_{vis}} = \sqrt{\left( \frac{ a }{
  \sqrt{E_{vis}/{\rm MeV}} } \right)^2
  +b^2},
 \label{eq:Eres}
 \end{align}
is composed of two parts.
The first term in the square-root represents the statistical uncertainty, and
the second one gives the systematic uncertainty~\cite{Eres}. The observed
antineutrino distribution by a detector with $N_p$ free protons after an exposure time $T$ can
then be expressed as
\begin{align}
 \frac{dN}{dE^{obs}_{vis}} =&
 \frac{N_pT}{4\pi L^2}\int^\infty_{E_{thr}} dE_{\nu}
 \frac{dN}{dE_{\nu}}
 P_{ee}(L,E_{\nu})\nn\\
&\times \sigma_{IBD}(E_{\nu}) \,G(E_{\nu}-0.8 {\rm MeV} -E^{obs}_{vis},\delta E_{vis}).
\label{Nobs}
\end{align}
\bibliographystyle{bib/model1a-num-names}
\bibliography{bib/neutrino,bib/neutrino_c}
\end{document}
