%#BIBTEX jbibtex intro
\documentclass[final,5p,times,twocolumn]{elsarticle}
%% if you use PostScript figures in your article
%% use the graphics package for simple commands
\usepackage{graphics}
%% or use the graphicx package for more complicated commands
%% \usepackage{graphicx}
%% or use the epsfig package if you prefer to use the old commands
%% \usepackage{epsfig}

%% The amssymb package provides various useful mathematical symbols
\usepackage{amssymb}
\usepackage{amsmath}
%% The amsthm package provides extended theorem environments
%% \usepackage{amsthm}

%% The lineno packages adds line numbers. Start line numbering with
%% \begin{linenumbers}, end it with \end{linenumbers}. Or switch it on
%% for the whole article with \linenumbers after \end{frontmatter}.
%% \usepackage{lineno}

\usepackage{latexsym}
\usepackage{supertabular}
\usepackage{slashed}
\usepackage{booktabs}
\usepackage{cases}
\usepackage{bigints}
\biboptions{sort&compress}
%% natbib.sty is loaded by default. However, natbib options can be
%% provided with \biboptions{...} command. Following options are
%% valid:

%%   round  -  round parentheses are used (default)
%%   square -  square brackets are used   [option]
%%   curly  -  curly braces are used      {option}
%%   angle  -  angle brackets are used    <option>
%%   semicolon  -  multiple citations separated by semi-colon
%%   colon  - same as semicolon, an earlier confusion
%%   comma  -  separated by comma
%%   numbers-  selects numerical citations
%%   super  -  numerical citations as superscripts
%%   sort   -  sorts multiple citations according to order in ref. list
%%   sort&compress   -  like sort, but also compresses numerical citations
%%   compress - compresses without sorting
%%
%% \biboptions{comma,round}

% \biboptions{}
%%% global definitions %%%%
\newcommand{\nn}{\nonumber}
\newcommand{\simlt}{\lower.5ex\hbox{$\; \buildrel < \over \sim \;$}}

%%% local definitions %%%
\newcommand{\dsol}{\Delta_{21}}
\newcommand{\ddsol}{2\Delta_{21}}
\newcommand{\drct}{\Delta_{31}}
\newcommand{\ddrct}{2\Delta_{31}}
\newcommand{\datm}{\Delta_{32}}
\newcommand{\ssol}{\sin^22\theta_{12}}
\newcommand{\srct}{\sin^22\theta_{13}}
\newcommand{\gwkty}{$\,{\rm GW_{th}}\cdot$kt$\cdot$year\,}
\newcommand{\dms}{\Delta m^2_{21}}
\newcommand{\dmr}{\Delta m^2_{31}}
\newcommand{\dcT}{\Delta\chi^2}
\newcommand{\dcTm}{(\Delta\chi^2)_{min}}
\newcommand{\cT}{\chi^2}
\newcommand{\ve}{\bar{\nu}_e}
\newcommand{\exposure}{${\rm 20 \,GW_{th}}$$\cdot$5kt(12\% free proton
weight fraction)$\cdot$5yrs}
%\journal{Physics Letters B}
\journal{arXiv:hep-ph}
%\journal{Preliminary Draft}

\begin{document}

\section{Introduction}
\label{intro}
Now that a large $\theta_{13}$ has been measured at Daya Bay~\cite{DayaBay} and
RENO~\cite{Ahn:2012nd} experiments accurately, neutrino physics enters a
new era. One of the next challenges is determination of the mass
hierarchy. 
%The next challenges are the determination of the mass
%hierarchy and the CP phase in the Maki-Nakagawa-Sakata (MNS)
%matrix~\cite{Maki:1962mu}.
% The determination of the mass hierarchy leads
%to deeper understanding of the lepton flavour mixing~\cite{Mohapatra:2006gs} and provides
%important basis for astrophysics~\cite{astro}. 
Many ideas have been proposed, such as long baseline
accelerator based neutrino oscillation~\cite{SuperBeam,Beijing,T2KK}, atmospheric neutrino~\cite{AtmosphericNeutrino}, supernova
neutrino~\cite{SuperNova}, neutrino-less double-beta
decay~\cite{DoubleBeta} and medium baseline reactor antineutrino experiments~\cite{Petcov:2001sy,Choubey:2003qx,Learned:2006wy,Zhan:2008id,Batygov:2008ku,Zhan:2009rs,Ghoshal:2010wt}.

Among them, the medium baseline reactor antineutrino experiment has 
stimulated various re-evaluations of its physics potential and sensitivity
recently. 
Some works utilize the Fourier transform
technique~\cite{Ciuffoli:2012iz,Ciuffoli:2012bs,Qian:2012xh}, first
discussed in
Refs.~\cite{Learned:2006wy,Batygov:2008ku,Zhan:2008id}, to
distinguish the mass hierarchy. The main advantage of this technique is
that the mass hierarchy can be determined without precise knowledge of
reactor antineutrino's spectrum, the absolute value of the large mass-squared difference $|\Delta m_{31}^2|$, 
and the energy scale of a detector. Although interesting and attractive, 
this technique is somewhat subtle to incorporate the uncertainties of
the mixing parameters and to estimate its sensitivity to
the mass hierarchy. On the other hand, some works adopt the $\chi^2$
analysis~\cite{Ghoshal:2010wt,Ghoshal:2012ju,Qian:2012xh} and new
measure based on Bayesian approach\cite{Qian:2012zn}. These
methods utilize all available information from experiments, and it is
straightforward to incorporate the uncertainties to evaluate
the sensitivity, providing robust and complementary
results to the Fourier technique. 

In this paper, we analyze 
the sensitivity of medium baseline reactor antineutrino
experiments to the mass hierarchy for the baseline length of $10$--$100$ km and
the energy resolution $(\delta E/E)^2 =
\left(a/\sqrt{E/{\rm MeV}}\right)^2 +b^2$ in the range $2\% < a < 6\%$
and $b < 1\%$ with
the standard $\cT$ analysis. The optimal baseline length and the expected
statistical uncertainties of neutrino parameters, $\ssol, \srct, \dms$ and $\dmr$, are also estimated.
\bibliographystyle{bib/model1a-num-names}
\bibliography{bib/neutrino,bib/neutrino_c}
\end{document}
