%#BIBTEX jbibtex chi2
\documentclass[final,5p,times,twocolumn]{elsarticle}
\usepackage{epsfig,amsmath,amssymb}
\usepackage{graphicx}
\usepackage{slashed}
\usepackage{wrapfig}
\usepackage{amssymb}
\usepackage{epstopdf}
\usepackage{titlesec}
\usepackage{longtable}
\usepackage{supertabular}
%\usepackage{hyperref}

\renewcommand{\floatpagefraction}{0.7}
\newcommand{\nn}{\nonumber}
\newcommand{\eps}{\epsilon}
\newcommand{\lam}{\lambda}
\newcommand{\sig}{\sigma}
\newcommand{\M}{{\cal M}}
\newcommand{\Mpl}{\overline{M}_{\rm Pl}}
\newcommand{\gld}{\tilde G}
\newcommand{\no}{\tilde{\chi}^0_1}
\newcommand{\se}{\tilde e}
\newcommand{\go}{\tilde g}
\newcommand{\sq}{\tilde q}

%\renewcommand{\thesection}{\thechapter.\arabic{section}}
%\renewcommand{\thesubsection}{\thesection.\arabic{subsection}}
\renewcommand{\baselinestretch}{1.2}
\makeatletter
    \renewcommand{\theequation}{%
    \thesection.\arabic{equation}}
    \@addtoreset{equation}{section}
  \makeatother
  %\setlength{\textheight}{20mm}
%\renewcommand{\thechapter}{\Roman{chapter}}

\setlength{\voffset}{-5em}
\setlength{\hoffset}{-2.5em}
\setlength{\textheight}{240mm}
\setlength{\textwidth}{150mm}

\title{\textbf{\Huge Generating multi-jet events with MadGraph}}
\author{\textbf{\LARGE Yoshitaro Takaesu\thanks{e-mail: takaesu@post.kek.jp} }}

\begin{document}
\section{The sensitivity to the mass hierarchy}
\label{chi2}
After obtaining the energy distribution of reactor antineutrinos, we
would like to estimate the sensitivity of determining the mass hierarchy
using the standard $\cT$
analysis~\cite{Choubey:2003qx,Batygov:2008ku,Ghoshal:2010wt,Ghoshal:2012ju,Qian:2012xh}. 

To set the stage, we introduce the $\cT$ function as
\begin{align}
 \cT = \cT_{para} +\cT_{sys} +\cT_{stat}.
\label{eq:chi2_func}
\end{align}
The first term summarizes the prior knowledge on mixing parameters. In
reactor antineutrino experiments, these are the mixing angles, $\ssol$ and $\srct$, and the two mass-square
differences, $\dms$ and $|\dmr|$, whose contributions look like,
\begin{align}
 \cT_{para} &= \left\{ \frac{ (\ssol)^{\,fit} -( \ssol )^{\,input} }{ \delta\ssol }
 \right\}^2 \nn\\
&+\left\{ \frac{ ( \srct )^{\,fit} -(\srct)^{\,input} }{ \delta\srct }
 \right\}^2 \nn\\
&+\left\{ \frac{ ( \dms )^{\,fit} -( \dms )^{\,input} }{ \delta\dms }  \right\}^2 \nn\\
&+\left\{ \frac{ ( |\dmr| )^{\,fit} -( |\dmr| )^{\,input} }{ \delta|\dmr| } \right\}^2.
\label{chi2para}
\end{align}
The input values $Y^{input}$ and their uncertainties $\delta Y$ are listed in
Table~\ref{tb:fitting_params}.

The reactor antineutrino flux, IBD cross section,
fiducial volume and weight fraction of free proton can all be combined into a single overall factor.
Consequently, their contributions to the $\chi^2$ function can be represented by a single term as,
\begin{align}
 \cT_{sys} = \left( \frac{f_{sys}^{\,fit}-f_{sys}^{\,input} }{\delta f_{sys}}  \right)^2,
\label{chi2sys}
\end{align} 
 where $f^{\,input}_{sys} = 1$, and $\delta f_{sys} = 0.03$.
\begin{table}
\scalebox{0.9}{
\begin{tabular}{cccccc}
\hline\hline\addlinespace[2pt] %need \usepackage{booktabs}
$Y$ & $\ssol$ & $\srct$ & $\dms \,{\rm eV}^2$ & $|\dmr| \,{\rm eV}^2$ & $f_{sys}$ \\[2pt]
\hline \addlinespace[2pt]
$Y^{input}$ & $0.857$ & $0.089$ & $7.50\times 10^{-5}$ & $2.32\times 10^{-3}$
		 & $1$\\
$\delta Y$ & $0.024$ & $0.005$ & $0.20\times 10^{-5}$ & $0.1\times
		 10^{-3}$ & 0.03 \\
\hline\hline
\end{tabular}\\
}
\hspace{2em}
\caption{The input values $Y^{input}$ and their uncertainties $\delta Y$ taken from Refs.~\cite{Beringer:1900zz,DayaBay}. The uncertainty of $\srct$ can be
 5\% or less after 3 years running of Daya Bay experiment~\cite{Error_dmm31}.}
\label{tb:fitting_params}
\end{table}

The third term in (\ref{eq:chi2_func}) represents 
the statistical fluctuation. When we introduce bining w.r.t. $E^{obs}_{vis}$, 
it looks like
\begin{align}
\cT_{stat} = \sum_i \left( \frac{ N_i^{\,fit} -N_i^{NH(IH)} }{
\sqrt{N_i^{NH(IH)}} }  \right)^2
\label{chi2stat}
\end{align}
with the summation running over all the bins. 
Here, $N_i^{NH(IH)}$ is the event number for the $i_{th}$ bin when 
the hierarchy is NH
(IH), while $ N_i^{fit}$ is the theoretical prediction of the event
number either with right or wrong mass hierarchy, calculated as
a function of the four model parameters and the normalization factor
$f_{sys}$, which are all varied under the
constraints of (\ref{chi2para}) and (\ref{chi2sys}). In this study we
prepare the data $N_i^{NH(IH)}$ by using eq.(\ref{Nobs}) with the
input values of the five parameters for each mass hierarchy. 

In the limit of
infinitely many events, the bin size can be reduced to zero, and the sum
(\ref{chi2stat}) can be replaced by an integral,
\begin{align}
\cT_{stat} \rightarrow \bigintss_{E_{min}}^{E_{max}} dE^{obs}_{vis}
\left\{ \frac{ 
   \left( \frac{dN}{ dE^{obs}_{vis} } \right)^{\,fit} 
  -\left( \frac{dN}{ dE^{obs}_{vis} } \right)^{ NH(IH) } 
}
{ 
  \sqrt{ 
         \left( \frac{dN}{ dE^{obs}_{vis} } \right)^{ NH(IH) } 
  } 
}  \right\}^2,
\label{chi2stat2}
\end{align}
where $E_{min} = 1.8~\mbox{MeV}$ and $E_{max} = 8~\mbox{MeV}$ are the lower and upper limits of the
observed energy used to evaluate the $\cT$ function, respectively.
Although a finite bin size is required for actual experiments, we
adopt this zero-bin-size limit as measure of the maximum sensitivity. We then define $\dcT$ as
\begin{align}
 \dcT = \cT -\cT_{min},
\end{align}
where $\chi^2_{min}$ is the minimum of $\chi^2$, which is obviously zero in our approximation of neglecting
statistical fluctuations in data, $N_i^{NH(IH)}$. When wrong mass hierarchy is
assumed in the fit, the minimum of $\Delta\chi^2$, $(\Delta\chi^2)_{min}$, will deviate from zero, and
the wrong mass hierarchy can be rejected with significance
$\sqrt{(\Delta\chi^2)_{min}}\footnote{Recent study~\cite{Qian:2012zn} argues that the
sensitivity estimation based on Gussian distribution can be
overestimated, but we will not pursue this possibility in this study.}$.
\bibliographystyle{bib2/general}
%\bibliographystyle{bib/model1a-num-names}
%\bibliographystyle{bib2/jhep}

\bibliography{bib2/reference,bib2/neutrino}
\end{document}

%%
%% End of file `elsarticle-template-1a-num.tex'.
% LocalWords:  binning
