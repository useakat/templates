%#BIBTEX jbibtex section
\documentclass[final,5p,times,twocolumn]{elsarticle}
%% if you use PostScript figures in your article
%% use the graphics package for simple commands
\usepackage{graphics}
%% or use the graphicx package for more complicated commands
%% \usepackage{graphicx}
%% or use the epsfig package if you prefer to use the old commands
%% \usepackage{epsfig}

%% The amssymb package provides various useful mathematical symbols
\usepackage{amssymb}
\usepackage{amsmath}
%% The amsthm package provides extended theorem environments
%% \usepackage{amsthm}

%% The lineno packages adds line numbers. Start line numbering with
%% \begin{linenumbers}, end it with \end{linenumbers}. Or switch it on
%% for the whole article with \linenumbers after \end{frontmatter}.
%% \usepackage{lineno}

\usepackage{latexsym}
\usepackage{supertabular}
\usepackage{slashed}
\usepackage{booktabs}
\usepackage{cases}
\usepackage{bigints}
\biboptions{sort&compress}
%% natbib.sty is loaded by default. However, natbib options can be
%% provided with \biboptions{...} command. Following options are
%% valid:

%%   round  -  round parentheses are used (default)
%%   square -  square brackets are used   [option]
%%   curly  -  curly braces are used      {option}
%%   angle  -  angle brackets are used    <option>
%%   semicolon  -  multiple citations separated by semi-colon
%%   colon  - same as semicolon, an earlier confusion
%%   comma  -  separated by comma
%%   numbers-  selects numerical citations
%%   super  -  numerical citations as superscripts
%%   sort   -  sorts multiple citations according to order in ref. list
%%   sort&compress   -  like sort, but also compresses numerical citations
%%   compress - compresses without sorting
%%
%% \biboptions{comma,round}

% \biboptions{}
%%% global definitions %%%%
\newcommand{\nn}{\nonumber}
\newcommand{\simlt}{\lower.5ex\hbox{$\; \buildrel < \over \sim \;$}}

%%% local definitions %%%
\newcommand{\dsol}{\Delta_{21}}
\newcommand{\ddsol}{2\Delta_{21}}
\newcommand{\drct}{\Delta_{31}}
\newcommand{\ddrct}{2\Delta_{31}}
\newcommand{\datm}{\Delta_{32}}
\newcommand{\ssol}{\sin^22\theta_{12}}
\newcommand{\srct}{\sin^22\theta_{13}}
\newcommand{\gwkty}{$\,{\rm GW_{th}}\cdot$kt$\cdot$year\,}
\newcommand{\dms}{\Delta m^2_{21}}
\newcommand{\dmr}{\Delta m^2_{31}}
\newcommand{\dcT}{\Delta\chi^2}
\newcommand{\dcTm}{(\Delta\chi^2)_{min}}
\newcommand{\cT}{\chi^2}
\newcommand{\ve}{\bar{\nu}_e}
\newcommand{\exposure}{${\rm 20 \,GW_{th}}$$\cdot$5kt(12\% free proton
weight fraction)$\cdot$5yrs}
%\journal{Physics Letters B}
\journal{arXiv:hep-ph}
%\journal{Preliminary Draft}

\begin{document}

\section{Reactor antineutrino flux}
\label{basic}
As well known, the vacuum $\bar{\nu_e}$ survival probability is expressed as
\begin{align}
 P_{ee} =& 1 -\cos^4\theta_{13}\ssol\sin^2\left(\dsol\right) \nn\\
& -\cos^2\theta_{12}\srct\sin^2\left(\drct\right) \nn\\
& -\sin^2\theta_{12}\srct\sin^2\left(\datm\right),
\label{pee1}
\end{align}
where $\theta_{ij}$ is a neutrino mixing angle. $\Delta_{ij}$ is defined as
\begin{equation}
 \Delta_{ij} \equiv \frac{\delta m^2_{ij}L}{4E_{\nu}}, 
 \hspace{1em}(\delta m^2_{ij} \equiv m^2_i -m^2_j)
\label{deltaij}
\end{equation}
where $m_i, E_{nu}$ and $L$ are the mass of the $i_{th}$ lightest
neutrino, the energy of the traveling electron antineutrino and the
flight distance of the antineutrino. We further convert this expression to
the one which manifests the dependence on the mass hierarchy more clearly;
\begin{align}
P_{ee} =& 1 -\cos^4\theta_{13}\ssol\sin^2\left(\dsol\right) \nn\\
& -\srct\sin^2\left(\drct\right) \nn\\
  &
 -\sin^2\theta_{12}\srct\cos\left(\ddrct\right)\sin^2\left(\dsol\right) \nn\\
& \pm \frac{\sin^2\theta_{12}}{2}\srct\sin\left(\ddrct\right)\sin\left(\ddsol\right).
\label{pee2}
\end{align}
The plus or minus sign in the last term corresponds to the normal hierarchy (NH) or
the inverted hierarchy (IH) cases, respectively. Only the last term
depends on the mass hierarchy (MH). From this expression, it is obvious
that the probability becomes most
  sensitive to the mass hierarchy when 
\begin{equation}
\ddsol = \frac{2n-1}{2}\pi, (n = 1,2,3,\cdots),
\label{condition1}
\end{equation}
and the MH difference vanishes when
\begin{equation}
\ddsol = n\pi, (n = 1,2,3,\cdots).
\label{condition2}
\end{equation}
For example, for $L=50$km, the condition (\ref{condition1}) and
(\ref{condition2}) with $n=1$
correspond to $E_{\nu}\sim 6$MeV and 3 MeV, respectively. These conditions are important for determining
MH using reactor neutrinos with intermediate baseline length, $L\sim
10-100$ km as we will see in the following discussions. \\

The energy distribution of $\nu_e$ with the energy $E_{\nu}$ MeV from a reactor complex with $P {\rm GW_{th}}$ can be
estimated as \cite{Zhan:2008id}
\begin{align}
 \phi(E_{\nu}) =& f_{\,{}^{235}U}\exp(0.870-0.160E_{\nu}-0.091E_{\nu}^2)
 \nn\\
 &+f_{\,{}^{239}P_u}\exp\left( 0.896 -0.239E_{\nu} -0.0981E_{\nu}^2 \right) \nn\\
 &+f_{\,{}^{238}U}\exp\left( 0.976 -0.162E_{\nu} -0.0790E_{\nu}^2 \right) \nn\\
 &+f_{\,{}^{241}P_u}\exp \left( 0.793 -0.080E_{\nu} -0.1085E_{\nu}^2
 \right) \left[ \frac{1}{{\rm MeV}} \right],
\end{align}
where $f_a$ denotes the averaged-fraction of the number of fission per
second for the radio-active isotope $a$ in the reactor fuel; $f_{\,{}^{235}U}=0.58,
\,f_{\,{}^{239}P_u}=0.30, \,f_{\,{}^{238}U}=0.07$ and
$\,f_{\,{}^{241}P_u}=0.05$. Here we assume that the reactor fuel consists of only above
four radio-active isotopes, and $\sum_a f_a =1$. Although these
fractions reflect the fraction of each isotopes in the reactor fuel and
variy over the running time of the reactors, we ignore this time
dependence and treat it as constants in this study. The number density
of reactor antineutrino with the energy $E_{\nu}$ MeV per second coming from reactors
with $P \,{\rm [GW_{th}]}$ thermal power is then
expressed as
\begin{align}
 \frac{dN_{RCT}}{dE_{\nu}} = \frac{P}{\sum_a f_a \epsilon_a}\sum_b
 f_b\phi_b(E_{\nu}) \times 6.24\cdot 10^{21} \left[ \frac{1}{{\rm s\cdot MeV}}
 \right],
\end{align}
where $\epsilon_a$ denotes the released energy in MeV per fission of the
isotope $a$. $6.24\cdot 10^{21}$ just comes from the unit
convention. 

The current reactor experiments such as Daya Bay\cite{An:2012eh},
RENO\cite{Ahn:2012nd} and Double Chooz\cite{Abe:2011fz}
uses free protons in their scintillation detector as targets for reactor
antineutrinos, and the interaction between the antineutrino and the
detector occurs via the inverse beta decay (IBD),
\begin{align}
 \bar{\nu_e} +p \rightarrow e^+ +n,
\label{IBD}
\end{align}
 where $p$ and $n$ denote the proton and the neutron. The tree-level total cross section
 of IBD is known as
 \begin{align}
  \sigma_{IBD} = 0.0952\left( \frac{E_e p_e}{\rm 1MeV^2} \right)\times 10^{-42} [{\rm cm^2}],
 \label{xsec}
 \end{align}
where $E_e$ and $p_e$ are the energy and momentum of the produced
positron. The threshold $\bar{\ne_e}$ energy ($E^{thr}_{\nu}$) for this
reaction is 
\begin{align}
 E^{thr}_{\nu} \sim& m_n -m_p +m_e \nn\\
\sim& 1.81 GeV,
\end{align}
 and the positron energy is related to $E_{\nu}$ as
\begin{align}
 E_e = E_{\nu} -1.3 [{\rm GeV}].
\end{align}

Therefore, the number of electron antineutrinos observed at a far
detector with $N_p$ target protons, the baseline of $L$ km and $T$ year
exposure reads
\begin{align}
 N =&
 \frac{N_{p}T4\pi}{L^2}\int^\infty_{E^{thr}_{\nu}}dE_{\nu}\int^\infty_0 dE^{'}_{\nu}
 \frac{dN_{RCT}}{dE^{'}_{\nu}}
 P_{ee}(L,E^{'}_{\nu}) \nn\\
&\times \sigma_{IBD}(E^{'}_{\nu}) \,G(E^{'}_{\nu}-E_{\nu},\delta E_{\nu}),
\end{align}
where $G(E^{'}_{\nu}-E_{\nu},\delta E)$ is a detector response
function which transforms the true neutrino energy $E^{'}_{\nu}$ to the
observed one, $E_{\nu}$, with the finite energy resolution of the
detector, $\delta E$; in this study we assume the 100\% efficiency of
the detector.


\bibliographystyle{bib/model1a-num-names}
\bibliography{bib/neutrino,bib/neutrino_c}
\end{document}