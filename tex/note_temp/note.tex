\documentclass[12pt,a4paper]{article}
\usepackage{amsmath,amssymb}
\usepackage{url,cite,slashed}
\usepackage[dvipdfmx]{graphicx}
\usepackage{subfigure}
\usepackage{color}

\setlength{\textwidth}{16.5cm}
\setlength{\textheight}{21.5cm}
\setlength{\oddsidemargin}{-.3 cm}
\setlength{\evensidemargin}{0cm}
\setlength{\topmargin}{0cm}
\setlength{\footskip}{1cm}

\makeatletter
    \renewcommand{\theequation}{%
    \thesection.\arabic{equation}}
    \@addtoreset{equation}{section}
\makeatother

\renewcommand{\topfraction}{1.0}
\renewcommand{\bottomfraction}{1.0}

\newcommand{\nn}{\nonumber}
\newcommand{\order}{{\mathcal O}}
\newcommand{\invfb}{{\rm fb}^{-1}}

%%%%%%%%%%%%%%%%%%%%%%%%%%%%%%%%%%%%%%%%%%%%%%%%%%%%
%% remove when submitting
\newcommand{\rem}[1]{\textcolor{red}{#1}}
%\newcommand{\rem}[1]{{$\spadesuit$\bf #1$\spadesuit$}}
%%%%%%%%%%%%%%%%%%%%%%%%%%%%%%%%%%%%%%%%%%%%%%%%%%%%

\begin{document}

\renewcommand{\thefootnote}{\fnsymbol{footnote}}
% \begin{titlepage}

% \begin{center}
% {\Large \bf 
% title
% }
% \end{center}

% \end{titlepage}

\setcounter{page}{1}
\renewcommand{\thefootnote}{\#\arabic{footnote}}
\setcounter{footnote}{0}

%%%%%%%%%%%%%%%%%%%%%%%%%%%%%%%%%%%%%%%%%%%%%%%%%%%%
\section{Check of our chi-square test statistics}
We investigate that how well the definition of the chi-square test
statistics bellow
\begin{align}
 \chi^2 = \sum_{i=1}^{N_{\rm bins}} \frac{(N^{\rm data}_i
 -\overline{N}_i)^2}{N^{\rm data}_i} 
\label{eq:ourchi2}
\end{align}
approximates the nominal definition
\begin{align}
 \chi^2 = \sum_{i=1}^{N_{\rm bins}} \frac{(N^{\rm data}_i
 -\overline{N}_i)^2}{\sigma_i},
\label{eq:chi2}
\end{align}
which obeys the $\chi^2$ distribution of $N_{\rm bins}$ degrees of
freedom, where $N^{\rm data}_i$ are assumed to distributes by the normal
distribution ${\cal N}(\overline{N}_i,\sigma_i)$.

We produced 0.1 million sets of pseudo-data, where each data set is
consisted with $N_{\rm bins}$ binned event numbers. We then calculate
the test statistics, Eq.~\eqref{eq:ourchi2}, for each data set and make
distribution of the test statistics. From the distributions, we estimate
the statistical significance, $\sigma$ for several values of the test
statistics. 

Figs. 1, 2 and 3 show the estimated statistical significances for
different $\overline{N}$ and $N_{\rm bins}$: Fig. 1, 2 and 3 corresponds to $\overline{N} = 1000, 100$ and 20,
respectively, and $N_{\rm bins}
= 1, 10 and 50$ from the left to the right panel in each Figure.
The horizontal axis shows the statistical significance expected for the
$\chi^2$ distribution of $N_{\rm bins}$ degrees of freedom, and the
vertical axis shows the estimated statistical significance. The orange
and blue dots show the results for the nominal definition of the $\chi^2$
test statistics, Eq.~\eqref{eq:chi2}, and our definition of it, Eq.~\eqref{eq:ourchi2},
respectively. The red line
shows the exact results of the $\chi^2$ distribution of $N_{\rm bins}$
degrees of freedom. 
\begin{figure}[ht]
 \centering
\resizebox{1.0\textwidth}{!}{
\includegraphics[width=1.0\textwidth]{figures/Nbar=1000.pdf}
}
  \caption{The estimated statistical significance as a function of the
 expected significance in the case of the $\chi^2$ distribution of
 $N_{\rm bins}$ d.o.f. $\overline{N} = 1000$. $N_{\rm bins} = 1, 10$ and
 50 from the left to the right panel. The orange
and blue dots show the results for the nominal definition of the $\chi^2$
test statistics, Eq.~\eqref{eq:chi2}, and our definition of it, Eq.~\eqref{eq:ourchi2},
respectively. The red line
shows the exact results for the $\chi^2$ distribution of $N_{\rm bins}$
degrees of freedom.}
\label{fig:nbar=1000}
\end{figure}

\begin{figure}[ht]
 \centering
\resizebox{1.0\textwidth}{!}{
\includegraphics[width=1.0\textwidth]{figures/Nbar=100.pdf}
}
  \caption{Same as Fig.~\ref{fig:nbar=1000}, but $\overline{N} = 100$.}
\label{fig:nbar=100}
\end{figure}

\begin{figure}[ht]
 \centering
\resizebox{1.0\textwidth}{!}{
\includegraphics[width=1.0\textwidth]{figures/Nbar=20.pdf}
}
  \caption{Same as Fig.~\ref{fig:nbar=1000}, but $\overline{N} = 20$.}
\label{fig:nbar=20}
\end{figure}

From the figures we see that our definition of the test
statistics approximate the nominal definition well only for $\overline{N}
> 1000$. Our definition may be used for rough estimation of the
statistical significance for $\overline{N}\sim \order(100)$. However,
for less $\overline{N}$, our definition gives significantly different
significances from the nominal
definition. This can be seen from the $\overline{N} = 20$ cases. 
The approximation gets worse for larger $N_{\rm bins}$ cases in all the
$\overline{N}$ cases. 

We conclude that our definition of the test statistics should not be
used for data sets where event numbers in each bin is at most $\order(10)$. 
%%%%%%%%%%%%%%%%%%%%%%%%%%%%%%%%%%%%%%%%%%%%%%%%%%%%
\subsection*{Acknowledgements}
This note is written in replying to the referee comments for our
manuscript ``Revisiting T2KK and T2KO physics potential and
$\nu_\mu$-$\bar{\nu}_\mu$ beam ratio''.

%%%%%%%%%%%%%%%%%%%%%%%%%%%%%%%%%%%%%%%%%%%%%%%%%%%%
%\bibliographystyle{../../draft/biblio/general}
%\bibliography{../../draft/biblio/reference}

\end{document}
